\begin{table}[htb]
    \def\aColWidth{0.2\textwidth}
    \def\bColWidth{0.15\textwidth}
    \def\cColWidth{0.24\textwidth}
    \def\dColWidth{0.14\textwidth}
    \def\eColWidth{0.16\textwidth}
    \caption{底部赝氢钝化的\cemb{InSb}切片模型上表面和次表面之间界面原子对的平均键长以及-ICOHP值}
    \label{tab:IS_ICOHP_pH}
    \setlength\tabcolsep{6pt}{\begin{tabular}{
        m{\aColWidth}
        m{\bColWidth}
        m{\cColWidth}
        >{\centering}m{\dColWidth}
        >{\centering\arraybackslash}m{\eColWidth}
        }
        \toprule
        结构模型&表面层构型&原子对&平均键长(\si{\angstrom})&平均-COHP值(\si{\electronvolt\per\pair})\\   
        \midrule
        \multirow{2.5}{*}{\shortstack[l]{底部赝氢钝化\\\cemb{InSb}切片模型}}&\cemb{Sb}极性&\cemb{In}(表面层)-\cemb{Sb}(块体)&2.91 &3.08\\
        \cmidrule(l){2-5}
        &\cemb{In}极性&\cemb{Sb}(表面层)-\cemb{In}(块体)&2.91 &3.17\\
        \bottomrule
    \end{tabular}}
\end{table}
