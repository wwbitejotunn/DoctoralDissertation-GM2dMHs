% LTeX: language=en-US
\begin{englishabstract}
    Two-dimensional materials and its heterostructures with excellent physical and chemical properties have broad application prospects and extraordinary research value in the fields of electronics and optoelectronics.

    Various 2D materials and 2D material heterostructures have been synthesized by different synthesis methods. However, due to the complex growth process and growth  mechanism of 2D materials, it is still difficult to synthesis large-area, high-quality 2D materials and two-dimensional heterostructures in a low-cost, fast, and highly controllable way. The first obstruction is the complex reaction mechanism in the growth process of two-dimensional materials. Small changes in  parameters such as substrate chemical reaction, gas-phase chemical reaction, and growth temperature will heavily affect the growth quality and morphology of two-dimensional materials and two-dimensional heterostructures. Secondly, the fine electronic structure of two-dimensional materials further improves the requirements for the growth quality and control precision level of two-dimensional materials and two-dimensional heterostructures.

    This dissertation aim to make better understanding of the growth process and provide theoretical reference for high-quality and highly controllable growth method of two-dimensional materials and its heterostructures. In this dissertation, growth process and mechanism of two-dimensional materials and its heterostructures were carefully investigated. Substrate, growth atmosphere and other growth parameters were studied and discussed for their influence of the growth process and morphology evolution of two-dimensional materials and its heterostructures. The corresponding growth mechanisms and growth model was addressed.

    Starting from graphene, this paper explored the substrate factors on the growth of graphene in chemical vapor deposition(CVD). The steps on the substrate surface can alter the nucleation behavior of graphene domain in early growth stage. Different from the flat Cu substrate, the graphene grow beside the steps on Cu substrate tends to spontaneously evolve to the most preferred growth orientation, which make the multipoint orientated growth of graphene on the polycrystalline Cu substrate possible. The multipoint orientated growth of graphene can greatly reduce the grain boundaries between graphene domains and realize the rapid growth of large-area graphene single crystals.
    
    % Dissertation
    %//TODO english abstract need to be done
    
    \englishkeyword{Growth Mechanism, Theoretical Calculation, Two-dimensional Materials, Two-dimensional Heterostructures}
\end{englishabstract}