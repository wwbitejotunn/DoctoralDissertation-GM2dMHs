% LTeX: language=en-US
\begin{englishabstract}
    Two-dimensional materials and its heterostructures (2DMs and 2DMHs) with excellent physical and chemical properties have broad application prospects and extraordinary research value in the fields of electronics and optoelectronics.

    Various 2DMs materials and 2DMHs have been synthesized by different synthesis methods. However, due to the complex growth process and growth  mechanism of 2DMs, it is still difficult to synthesis large-area, high-quality 2DMs and 2DMHs in a low-cost, fast, and highly controllable way. The first obstruction is the complex reaction mechanism in the growth process of 2DMs. Small changes in  parameters such as substrate chemical reaction, gas-phase chemical reaction, and growth temperature will heavily affect the growth quality and morphology of 2DMs and 2DMHs. Secondly, the fine electronic structure of 2DMs further improves the requirements for the growth quality and control precision level of 2DMs and 2DMHs.

    This dissertation aim to make better understanding of the growth process and provide theoretical reference for high-quality and highly controllable growth method of 2DMs and its heterostructures. In this dissertation, growth process and mechanism of 2DMs and its heterostructures were carefully investigated. Substrate, growth atmosphere and other growth parameters were studied and discussed for their influence of the growth process and morphology evolution of 2DMs and its heterostructures. The corresponding growth mechanisms and growth model was addressed.

    Starting from graphene, this dissertation explored the substrate factors on the growth of graphene in chemical vapor deposition (CVD). The steps on the substrate surface can alter the nucleation behavior of graphene domain in early growth stage. Different from the flat Cu substrate, the graphene grow beside the steps on Cu substrate tends to spontaneously evolve to the most preferred growth orientation, which make the multipoint orientated growth of graphene on the polycrystalline Cu substrate possible. The multipoint orientated growth of graphene can greatly reduce the grain boundaries between graphene domains and realize the rapid growth of large-area graphene single crystals. One step further, this dissertation explored the effect of chemical atmosphere on the growth of CVD graphene. The non-uniform promotion effect of oxygen on growth and etching behavior of few layers graphene(FLG) were investigated. The competition between oxygen-promoted growth and etching introduce the growth mode switch in FLG. A numerical model were developed to describe the non-uniform promotion effect of oxygen on both growth and etching of FLG and a corresponding growth model phase diagram were established. The study of graphene growth mechanism provide a theoretical insight for large-area single-crystal graphene growth and the layer number tuning of FLG.
    
    Next, this dissertation investigated the growth mechanism of non-planar polar 2DMs which have more complex structure than planar ones. A systematic study were carried out to explore the growth process and polarity evolution of polar bilayer \ce{InSb(111)} on \ce{Bi(001)} substrate from adatoms through amorphous monolayer to \ce{In}-polarity bilayer \ce{InSb}. Calculations revealed the amorphous configuration of monolayer \ce{InSb}, fully investigate the structural evolution of the self-polarization process of bilayer \ce{InSb}. Phase diagrams were developed to describe the ongoing polarization process of bilayer InSb under various chemical environments as a function of deposition time. The growth mechanism and polarity phase diagram of bilayer InSb can advance the progress of polarity controllable growth of low-dimensional III-V compound semiconductors.

    Finally, based on the understanding of 2DMs growth mechanism, this dissertation extend the growth mechanism investigation to 2DMHs including vertical stacking graphene/ \ce{h-BN} heterostructures and lateral binding graphene/\ce{VSe2} heterostructures. for graphene/ \ce{h-BN} heterostructures, this dissertation developed a proximity catalysis route by introducing Cu vapor to accelerate the direct stacking growth of graphene on \ce{h-BN}. For Cu vapor, this dissertation prove the availability near the surface of \ce{h-BN} and catalytic activity of \ce{CH4} decomposition. Further study revealed the growth sequence of direct growth graphene on \ce{h-BN}, provide structrue evolution from line-like cluster through ring-like cluster to graphene-like hexagon cluster. For the graphene/\ce{VSe2} lateral heterostructures, this dissertation found that the growth of it required highly active \ce{V} and \ce{Se} radicals for adatoms adsorption and nucleation of heterostructures growth. Further investigate on the growth mechanism from the perspectives of thermodynamics and kinetics discovered the selective growth of monolayer \ce{VSe2} on the edge of bilayer graphene steps. 
    The exploration of the growth mechanism of the 2DMHs provides a fast and efficient growth method for the growth of graphene/ \ce{h-BN} vertical 2DMs without introducing additional impurities. The investigation of graphene/\ce{VSe2} lateral 2DMHs provides the limiting conditions for the growth of \ce{VSe2} on graphene, and the selective growth mechanism provides a new pathway for the spatially controllable growth of graphene/\ce{VSe2} lateral 2DMHs. 

    The theoretical growth mechanism research in this dissertation is aim to provide theoretical insight and effective theoretical model for higher quality, lower cost, more controlable approachs of the synthesis of 2DMs and 2DMHs and finally promote the practicality and industrialization of 2DMs and 2DMHs.
    
    \englishkeyword{Growth Mechanism, Theoretical Calculation, Two-dimensional Materials,\\ Two-dimensional Heterostructures}
\end{englishabstract}