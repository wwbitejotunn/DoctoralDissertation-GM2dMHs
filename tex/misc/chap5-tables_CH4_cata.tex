\begin{table}[tb]
    \def\aColWidth{0.20\textwidth}
    \def\bColWidth{0.10\textwidth}
    \def\cColWidth{0.10\textwidth}
    \def\dColWidth{0.10\textwidth}
    \def\eColWidth{0.10\textwidth}
    \def\fColWidth{0.10\textwidth}
    \def\abColwidth{0.3\textwidth}
    \caption{\cemb{Cu}团簇和\cemb{Cu(111)}衬底表面\cemb{CH4}脱氢裂解反应的激活能$\energyVar{a}{}$和反应能$\Delta\energyVar{}{}$}
    \label{tab:CH4_cata}
    \begin{tabular}{
        m{\aColWidth}
        >{\centering}m{\bColWidth}
        >{\centering}m{\cColWidth}
        >{\centering}m{\dColWidth}
        >{\centering}m{\eColWidth}
        >{\centering\arraybackslash}m{\fColWidth}
    }
    \toprule
    \multicolumn{2}{c}{\multirow{3.5}{*}{反应步骤}}&\multicolumn{3}{c}{\cemb{Cu}团簇构型}&\cemb{Cu}衬底表面\\
    \cmidrule{3-6}
    &&\cemb{Cu}团簇&\cemb{Cu2}团簇&\cemb{Cu3}团簇&\cemb{Cu(111)}表面\citing{RN789-2013}\\
    \midrule
    \multirow{2}{\abColwidth}{\cemb{CH4 -> CH3 + H}} & $\energyVar{a}{}\ \ (\si{\electronvolt})$&1.51&1.57&0.92&1.88\\
    &$\Delta \energyVar{}{}\;(\si{\electronvolt})$&0.43&0.35&0.05&0.86\\
    \multirow{2}{\abColwidth}{\cemb{CH3 -> CH2 + H}} &$\energyVar{a}{}\ \ (\si{\electronvolt})$&2.52&1.61&1.54&1.47\\
    &$\Delta \energyVar{}{}\;(\si{\electronvolt})$&2.34&0.76&1.53&1.05\\
    \multirow{2}{\abColwidth}{\cemb{CH2 -> CH + H}} &$\energyVar{a}{}\ \ (\si{\electronvolt})$&2.37&2.70&1.53&1.05\\
    &$\Delta \energyVar{}{}\;(\si{\electronvolt})$&2.15&2.70&1.05&0.32\\
    \multirow{2}{\abColwidth}{\cemb{CH -> C + H}} &$\energyVar{a}{}\ \ (\si{\electronvolt})$&2.25&1.53&1.23&2.21\\
    &$\Delta \energyVar{}{}\;(\si{\electronvolt})$&2.25&1.38&1.23&1.20\\
    \midrule
    \multirow{2}{\abColwidth}{\shortstack{总反应\\\cemb{CH4 -> C + 4H}}}&$\energyVar{a}{}\ \ (\si{\electronvolt})$&2.52&2.70&1.54&2.21\\
    &$\Delta \energyVar{}{}\;(\si{\electronvolt})$&7.17&5.19&3.09&3.08\\
    \bottomrule
    \end{tabular}
\end{table}