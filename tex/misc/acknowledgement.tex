\thesisacknowledgement
\IfSubStr{\MYANONYMOUS}{Y}{
}{
    从踏入大学校门算起,已经在电子科技大学生活,学习,研究了近九年时光。虽然一直在抱怨成都没有太阳的天空、无从下口的辣味美食、口罩不离的空气质量。但从18岁至27岁,九年的时光让成都成为了我的第二故乡。电子科技大学作为究极的理工科院校,在九年漫长的理工科学习时光之中,我仍旧感受到了在冰冷的公式和定理推论的缝隙之中外溢的人文情怀。那是正如“求真求实大气大为”的校训所表述的,科研工作者、工程师所代表的,在公式和定理框架之下的,对于真理和极限不断追求的热忱,以及对于认识世界的好奇和改变世界的勇气。是好奇和勇气带领我一直在攻读博士的道路上砥砺前行。
    
    在学业的路上,我首先要感谢我的导师牛晓滨教授。不知不觉间在牛老师的门下已经学习和研究超过了五年时间,牛老师学识渊博,理论嗅觉敏锐,给予我们最大的探索的自由和尽可能多的资源支持。在整个攻读博士的生涯中,每篇论文每个工作都在和牛老师的激烈探讨中完成,牛老师独到的理论见解,清晰的物理图像,严谨的治学态度让我们深感敬佩。与学术上的严谨和苛求相对应的是牛老师对我们生活上的关系,情感上的关注和思想上的关怀,态度的反差体现的是牛老师追求真理的固执和立德树人的教育理念。感谢牛老师辛苦的培养和指导。

    我要感谢与我合作的实验研究者,包括李雪松教授、王丽平研究员、胡文成教授、熊杰教授、李含冬教授以及沈长青同学、姚梦麒同学、蒋澄同学、杨晖同学、侯雨婷同学。和实验研究者的合作犹如拔河,但目的却是把碗口粗的绳索生生拔断,并且分析断口的形态。每次讨论会上的争论,对于模型偏差的异议,对于废弃数据的懊恼,对于研究计划的改进,对于模型吻合的兴奋,都使得我们对在认知世界,探索世界的路上迈出了更为坚实的一步。与实验研究者的同行让我对实验有了更深理解和敬畏,感谢各位的辛苦工作。
    
    感谢ICAM课题组中的其他老师和同学,感谢王建伟老师对我学术上的建议和指导。感谢张生辉同学这几年的同甘共苦。感谢胡杰师兄、赵旭红师妹、魏川茗师弟、张坦师弟,感谢各位几年来对我的关照。

    感谢刘晓教授,虽然只在您的门下学习了一年的量子场论课程,但场论的学习让我触摸到了一个完全不同的粒子物理世界,您对知识的渴求和对真理的执着是我不断追求的榜样。

    感谢罗林霏的陪伴和支持,有幸在平淡的学术生涯中遇见。因为有你,在任何可能的事情上我都想要比当下做的更好,同时让我更有勇气去尝试完全没有涉及的领域。作为学术上的师兄,好胜心让我尽可能的维持技术上的领先。但机智的你总能在不经意间给予我致命一击,彻底破坏我想要塑造的万能设定。作为生活中的同伴,万分感谢你对比较混沌的我的各种理解包容,以后还敢。

    感谢我的父母和家人,都说攻读博士的道路是艰辛的。感谢父母的理解和鼓励,家人作为我永远的后盾,让原本波折的博士道路变得和缓。

    学业的结束是事业的开始,希望求知的好奇不要被未来的琐事所消耗,希望求索的勇气不要被未来的困难所打倒。希望我能以微薄的学识和无限的努力,在某一领域迈出毫厘的一步,为社会做出微小的贡献,为人类提升细微的可能性。
    %//TODO 致谢
}
