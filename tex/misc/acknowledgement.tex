\thesisacknowledgement
\IfSubStr{\MYANONYMOUS}{Y}{
}{
    从踏入大学校门算起,已经在电子科技大学生活,学习,研究了近九年时光。虽然一直在抱怨成都没有太阳的天空、无从下口的辣味美食、口罩不离的空气质量。但从18岁至27岁,九年的时光让成都成为了我的第二故乡。电子科技大学作为究极的理工科院校,在九年漫长的理工科学习时光之中,我仍旧感受到了在冰冷的公式和定理推论的缝隙之中外溢的人文情怀。那是正如“求真求实大气大为”的校训所表述的,科研工作者、工程师所代表的,在公式和定理框架之下的,对于真理和极限不断追求的热忱,以及对于认识世界的好奇和改变世界的勇气。是好奇和勇气带领我一直在攻读博士的道路上砥砺前行。
    
    在学业的路上,我首先要感谢我的导师牛晓滨教授。不知不觉间在牛老师的门下已经学习和研究超过了五年时间,牛老师学识渊博,理论嗅觉敏锐,对待学生宽容,给予学生最大的探索的自由和最多的资源支持。在整个攻读博士的生涯中,我的每篇论文都在和牛老师的角力之中完成,牛老师独到的理论见解,清晰的物理图像,严谨的治学态度让学生深感敬佩。 %//TODO

    我要感谢与我合作的实验研究者,包括李雪松教授、王丽平研究员、胡文成教授、熊杰教授以及沈长青同学、姚梦麒同学、蒋澄同学、杨晖、侯雨婷同学。和实验研究者的合作犹如拔河,但目的却是把碗口粗的绳索生生拔断,并且分析断口的形态。每次讨论会上的争论,对于模型偏差的异议,对于废弃数据的懊恼,对于研究计划的改进,对于模型吻合的兴奋,都使得我们对在认知世界,探索世界的路上卖出了更为坚实的一步。与实验研究者的同行让我对实验有了更深理解和敬畏,感谢各位的辛苦工作。
    
    感谢ICAM课题组中的其他老师和同学, %//TODO

    感谢刘晓教授 %//TODO

    感谢罗林霏学妹,%//TODO

    感谢我的父母和家人,都说攻读博士的道路是艰巨的。感谢父母的理解和鼓励,家人作为我永远的后盾,让原本波折的博士道路变得和缓。

    学业的结束是事业的开始,希望求知的好奇不要被未来的琐事所消耗,希望求索的勇气不要被未来的困难所打倒。希望我能以微薄的学识和无限的努力,在某一领域迈出毫厘的一步,为社会做出微小的贡献,为人类提升细微的可能性。
    %//TODO 致谢
}
