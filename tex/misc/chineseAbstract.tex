% LTeX: language=zh-CN
\begin{chineseabstract}
    具有优异的物理化学性质的二维材料及其异质结在电子,光电子等领域具有广阔的应用前景和不凡的研究价值。迄今为止,二维材料及其异质结在新型电子和光电子器件的研究中得到了大量的关注。大量隶属于不同材料体系的二维材料及其异质结被不同的实验手段制备合成。但是,由于二维材料复杂的生长过程和生长动力学机制,低成本、快速、形貌高可控地大面积制备高质量的二维材料和二维材料异质结仍面临许多挑战。首先是二维材料及其异质结生长过程中复杂的反应机理,制备过程中生长参数细微的变化都会改变衬底表面和气相环境中的化学反应过程,从而影响二维材料及其异质结的生长形貌和制备质量。其次是二维材料精细的电子结构进一步提高了对二维材料及其异质结的制备质量和调控精确度的要求。
    
    为了能够更好地理解二维材料及其异质结的生长过程,为二维材料及其异质结的高质量可控生长提供理论基础,本论文对以石墨烯和双层\cemb{InSb}为代表的二维材料以及石墨烯基的二维材料异质结进行了深入的机理研究,讨论了包括衬底表面结构和生长气氛对于二维材料及其异质结生长过程和形貌演化的影响,建立了相应的生长模型。

    本论文首先从石墨烯出发,探究在化学气相沉积法中衬底对于单层石墨烯生长晶向的影响。研究发现衬底表面的台阶会改变石墨烯的成核机制。与平坦衬底上生长的石墨烯不同,在台阶边缘生长的石墨烯会自发的向优先生长方向进行结构演化,实现石墨烯在多晶铜衬底表面的多点自发定向生长。多点自发定向生长的石墨烯能够大幅减少石墨烯晶畴之间的晶界,实现大面积石墨烯单晶的快速生长。随后,本论文进一步探究了气相沉积法生长石墨烯过程中化学气氛对于多层石墨烯的生长作用。深入研究了气氛中的氧对于多层石墨烯的生长和蚀刻行为的非均一促进机制。通过对氧促生长和氧促蚀刻之间竞争关系的进一步探究,建立了氧通量调控的多层石墨烯生长和蚀刻模式切换模型,绘制了相应的生长模式切换相图。对于石墨烯生长机理的研究为石墨烯单层大面积单晶生长以及石墨烯多层调控提供了理论基础。

    接着,本论文将研究扩展到具有结构极性的非平面二维材料。相比于平面二维材料,非平面极性二维材料具有更复杂晶体结构和生长机理。通过对非平面二维材料\cemb{InSb(111)}在\cemb{Bi(001)}衬底表面进行系统的分析计算,本论文探究了双层\cemb{InSb}从原子吸附到非晶态单层生长再到双层\cemb{In}极性自极化的整个生长过程和极性演化规律,揭示了\cemb{Bi}衬底表面单原子层\cemb{InSb}的非晶态构型,探究了双层\cemb{InSb}由非晶态构型极化至\cemb{In}极性过程的结构演化过程并且绘制了双层\cemb{InSb}的生长极性演化相图。最后,通过将表面能和界面能解耦的方式,本论文深入探讨了这两种能量在单层\cemb{InSb}生长以及双层\cemb{InSb}自极化过程中的对于原子形貌的作用和竞争机制。研究发现非晶态单层\cemb{InSb}来源于\cemb{Bi}衬底对\cemb{InSb}强烈的相互作用,而双层\cemb{InSb}的极化则归功于二层重构\cemb{InSb}表面能的大幅下降。对于双层\cemb{InSb(111)}生长机理和极性演化机制的研究有利于加深我们对低维III-V化合物半导体纳米结构生长过程的理解,帮助我们其生长过程和生长形貌进行更有效地控制。

    最后,本论文在二维材料生长机理研究的基础上对纵向堆叠的石墨烯/\cembNHS{h-BN}二维材料异质结和横向拼接的石墨烯/\cembNHS{VSe2}二维材料异质结的生长机理进行了探索。提出并证明了利用\cemb{Cu}蒸气在\cemb{h-BN}表面催化裂解\cemb{CH4},加速生长石墨烯/\cembNHS{h-BN}纵向二维异质结的方法及可行性。探究了石墨烯在\cemb{h-BN}表面的生长序列,给出了石墨烯早期生长阶段从线性团簇到环形团簇到六边形团簇的形貌演化过程。对石墨烯/\cembNHS{VSe2}横向二维异质结生长机理的研究发现石墨烯/\cembNHS{VSe2}横向异质结的生长需要高活性的\cemb{V}和\cemb{Se}自由基作为生长前驱体。同时考虑热力学和动力学的生长因素,本论文发现了单层\cemb{VSe2}在双层石墨烯台阶边缘的选择性生长机制。对于二维材料异质结生长机制的探究为石墨烯/\cembNHS{h-BN}纵向二维异质结的生长提供了快速高效且不引入额外杂质的生长方式;对石墨烯/\cembNHS{VSe2}横向二维异质结生长机理的研究给出了异质结生长的限制条件,\cemb{VSe2}的选择性生长为异质结的空间位置可控生长提供了新的思路。

    本论文对于二维材料及其异质结的生长机理的研究为更低成本、更高质量、形貌调控更精确的二维材料及其异质结的生长方式提供了丰富的理论基础和有效的理论模型,有利于进一步推进二维材料及其异质结的实用化水平。

    \chinesekeyword{生长机理,理论计算,二维材料,二维材料异质结}
\end{chineseabstract}