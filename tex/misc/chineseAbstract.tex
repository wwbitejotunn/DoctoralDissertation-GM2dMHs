% LTeX: language=zh-CN
\begin{chineseabstract}
    具有优异的物理化学性质的二维材料及二维材料异质结在电子,光电子等领域具有广阔的应用前景和不凡的研究价值。迄今为止,二维材料及二维材料异质结在新型电子、光电子器件的研究中得到了大量的关注。各种隶属于不同材料体系的二维材料及二维材料异质结被通过不同的制备手段合成而出。但是,由于二维材料复杂的生长过程和生长动力学机制,想要低成本、快速、形貌高可控地生长制备出大面积、高质量的二维材料及二维材料异质结仍旧困难重重。首先是二维材料及二维材料异质结生长过程中复杂的反应机理,衬底化学反应、气相化学反应,生长温度等各种微小参数的变化都会影响二维材料及二维材料异质结的生长质量和形貌调控可行性。其次是二维材料精细的电子结构进一步提高了对二维材料及二维材料异质结生长质量和调控精确等级的要求。
    
    为了能够更好地理解二维材料及二维材料异质结的生长过程,为二维材料及二维材料异质结的高质量可控生长提供理论基础,本论文对多种二维材料以及二维材料异质结进行了深入的机理研究,讨论了包括衬底、生长气氛在内的多种生长参数对于二维材料及二维材料异质结生长过程和形貌演化的影响,并且建立了相应的生长模型。

    本论文从石墨烯出发,首先探究了在化学气相沉积法中衬底对于单层石墨烯生长的影响。研究发现衬底表面的台阶会改变石墨烯的成核机制。与平坦衬底上生长的石墨烯不同,在台阶边缘生长的石墨烯会自发的向优先生长方向进行结构演化,实现石墨烯在多晶铜衬底表面的多点自发定向生长。多点自发定向生长的石墨烯能够大幅减少石墨烯晶畴之间的晶界,实现大面积石墨烯单晶的快速生长。随后,本论文进一步探究了气相沉积法生长石墨烯过程中化学气氛对于多层石墨烯的生长作用。深入研究了气氛中的氧对于多层石墨烯的生长和蚀刻行为的非均一的促进机制。通过对氧促生长、氧促蚀刻之间竞争关系的进一步探究,建立了氧辅助多层石墨烯生长、蚀刻模式切换模型,绘制了多层石墨烯的生长、蚀刻模式切换相图。对于石墨烯生长机理的研究为石墨烯单层大面积单晶生长以及石墨烯多层调控提供了理论基础。

    接着,本论文将目光从石墨烯这种平面非极性二维材料转向更为复杂的非平面极性二维材料的生长机理研究。系统的研究了非平面极性二维材料\cemb{InSb(111)}在\cemb{Bi(001)}衬底表面的从原子吸附到非晶态单层生长再到双层\cemb{In}极性自极化的整个生长过程和极性演化规律。揭示了\cemb{Bi}衬底表面单原子层\cemb{InSb}的非晶态构型。探究了双层\cemb{InSb}由非晶态构型极化至\cemb{In}极性过程的结构演化过程,绘制了双层\cemb{InSb}的生长极性演化相图。最后,通过将表面能和界面能解耦的方式,深入探究了二者在单层\cemb{InSb}生长、双层\cemb{InSb}极化过程的作用和竞争机制。发现了非晶态单层\cemb{InSb}来源于\cemb{Bi}衬底对\cemb{InSb}强烈的相互作用,而双层\cemb{InSb}的极化则归功于二层重构\cemb{InSb}表面能的大幅下降。对于双层\cemb{InSb(111)}生长机理和极性演化机制的研究有利于加深我们对低维III-V化合物半导体纳米结构生长过程的理解,帮助我们对低维III-V化合物半导体纳米结构的生长过程和生长形貌进行更有效的控制。

    最后,本论文在二维材料生长机理研究的基础上,对纵向堆叠的石墨烯/\cembNHS{h-BN}二维材料异质结和横向拼接的石墨烯/\cembNHS{VSe2}二维材料异质结的生长机理进行了探索。提出并证明了利用\cemb{Cu}蒸气在\cemb{h-BN}表面催化裂解\cemb{CH4},加速生长石墨烯/\cembNHS{h-BN}纵向二维异质结的方法及可行性。探究了石墨烯在\cemb{h-BN}表面的生长序列,给出了石墨烯早期生长阶段从线性团簇到环形团簇到六边形团簇的形貌演化过程。对石墨烯/\cembNHS{VSe2}横向二维异质结的生长机理的研究发现石墨烯/\cembNHS{VSe2}横向异质结的生长需要高活性的\cemb{V}、\cemb{Se}自由基作为前驱体进行吸附、成核。接着,本论文从热力学和动力学的角度探究了单层\cemb{VSe2}在双层石墨烯台阶边缘的选择性生长机制。对于二维材料异质结生长机制的探究为石墨烯/\cembNHS{h-BN}纵向二维异质结的生长提供了快速高效且不引入额外杂质的生长方式;对石墨烯/\cembNHS{VSe2}横向二维异质结生长机理的研究给出了异质结生长的限制条件,\cemb{VSe2}的选择性生长为异质结的空间位置可控生长提供了新的思路。

    本论文对于二维材料及二维材料异质结的生长机理的研究为更低成本、更高质量、形貌调控更精确的二维材料及二维材料异质结的生长方式提供了丰富的理论基础和有效的理论模型,有利于进一步推进二维材料及二维材料异质结的实用化水平。

    \chinesekeyword{生长机理,理论计算,二维材料,二维材料异质结}
\end{chineseabstract}