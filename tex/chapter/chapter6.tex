% LTeX: language=zh-CN

\chapter{总结与展望}
\section{总结}
针对二维材料及其异质结的生长机理,本论文分别对二维材料以及二维材料组成的异质结进行了生长过程的探究以及生长机理的分析,深入讨论了衬底、气氛、生长时间、外缘催化剂、生长温度等因素对于二维材料及二维材料异质结生长过程和形貌调控的作用和影响,为二维材料及其异质结的高质量可控生长提供了理论基础和机理解释。本论文的研究分为两个大方向,分别为二维材料的生长机理和二维材料异质结的生长机理。

对于二维材料的生长机理,本论文从经典的二维材料石墨烯入手,深入探究了化学气相沉积生长过程中衬底和生长气氛对于单层石墨烯定向生长和多层石墨烯生长模式切换的作用机理。关于衬底在生长过程中对石墨烯的影响,本论文证明了衬底表面的台阶破坏了原衬底表面的对称性,为石墨烯的成核生长提供了优先生长位点和优先生长方向,使得石墨烯能够进行定向生长。定向生长的石墨烯能够减少多点成核石墨烯之间的晶界,实现大规模石墨烯单晶的快速生成。关于生长气氛对于层石墨烯生长模式切换的作用机理,本论文揭示了气氛中氧气的存在突破了原有\cemb{Cu}衬底上石墨烯生长的自限制作用。氧对石墨烯多层的蚀刻和穿透作用的相互竞争导致了石墨烯多层交替出现蚀刻和生长的现象。通过对氧辅助多层石墨烯生长、蚀刻作用进行建模,给出了氧辅助多层石墨烯生长、蚀刻模式切换模型并且绘制了相应的生长模式切换相图。氧辅助多层石墨烯生长模式的切换有利于实现石墨烯生长过程中的层数调控。

随后,本论文将二维材料的生长机理的研究由平面非极性二维材料石墨烯拓展至非平面极性二维材料\cemb{InSb},系统地探究了\cemb{Bi(001)}衬底表面双层\cemb{InSb(111)}的生长机理和极性演化过程。研究发现单层\cemb{InSb(111)}表现出以多种混合极性平衡而成的非晶态而生长的第二层\cemb{InSb(111)}能够将双层\cemb{InSb}自发极化至\cemb{In}极性。针对双层\cemb{InSb}的极化过程,本论文给出了双层\cemb{InSb}在不同\cemb{In}原子化学势条件下包含不同沉积原子数的生长相图。同时,本论文通过赝氢饱和法和四面体法将表面能和界面能分离,发现单层\cemb{InSb}的非晶态结构源自衬底的强烈作用。而双层\cemb{InSb}中大幅下降的表面能使得单一极性的构型更具能量优势。具有低能态的表面重构结构通过双层\cemb{InSb}内部的界面作用,将第一层\cemb{InSb}转变为\cemb{InSb}极性的构型。双层\cemb{InSb}生长极性相图和极性演化机理的研究有助于加深对低维III-V化合物半导体纳米结构生长过程的理解,同时帮助对低维III-V化合物半导体纳米结构的生长过程和生长形貌进行更有效的控制。

在二维材料生长机理研究的基础上,本论文进一步探究了石墨烯/\cembNHS{h-BN}纵向堆叠而成的二维纵向异质结和石墨烯/\cemb{VSe2}横向拼接而成的二维横向异质结的生长机理。关于石墨烯/\cembNHS{h-BN}二维纵向异质结,本论文构建了一种利用\cemb{Cu}蒸气近邻催化效应在\cemb{h-BN}表面直接堆叠生长石墨烯的方法。通过结合流体力学计算和第一性原理计算,本论文对了\cemb{Cu}蒸气蒸气近邻催化\cemb{CH4}裂解的可行性进行了证明,同时探究了活性\cemb{C}原子在\cemb{h-BN}表面由单个吸附的\cemb{C}原子成核生长成\cemb{C24}团簇的生长演化序列。利用\cemb{Cu}蒸气近邻催化效应在\cemb{h-BN}表面直接堆叠生长石墨烯的方法能够在不引入额外杂质的情况下大幅提高石墨烯在\cemb{h-BN}表面的生长速率,提升石墨烯/\cembNHS{h-BN}二维纵向异质结的产率,推进基于石墨烯/\cembNHS{h-BN}二维纵向异质结的电子器件的产业化水平。随后,本论文对石墨烯/\cembNHS{VSe2}横向异质结的生长机理进行了研究。计算表明证明对于石墨烯/\cembNHS{VSe2}横向异质结的生长需要在生长环境中利用高温等方式进一步提升\cemb{V}原子和\cemb{Se}原子的能级水平,使其在气相中裂解成为气态的\cemb{V}自由基和\cemb{Se}自由基。同时,本论文发现了单层\cemb{VSe2}具有在双层石墨烯台阶边缘的选择性生长的特点。在热力学和动力学双重因素的驱使下,单层\cemb{VSe2}倾向于在双层的石墨烯台阶边缘以锯齿形匹配的方式形成横向异质结。双层石墨烯台阶锯齿边缘\cemb{VSe2}的选择性生长机理有利于实现石墨烯/\cembNHS{VSe2}组成的二维横向异质结的生长位置调控,进一步异质结的生长均匀性。

对于上述工作,本论文具有以下创新点\chinesecolon
\begin{enumerate}[labelsep=0em,label=(\arabic*),wide]
    \item 结合气相反应动力学模拟和第一性原理方法,建立氧辅助石墨烯多层生长/蚀刻模式变换模型,为石墨烯生长过程中的层数调控提供理论指导。
    \item 对\cemb{Bi}衬底表面双层\cemb{InSb}生长过程和极性演化进行了系统的研究,发现单层\cemb{InSb}非晶态的结构,探究了二层InSb对首层的极化作用,绘制了双层\cemb{InSb}的极性演化相图。
    \item 通过赝氢饱和法和四面体法,对\cemb{InSb}生长和极性演化过程中表面能和界面能的作用进行了解耦,证明双层\cemb{InSb}中二层对首层的极化作用来源于表面能的下降并通过\cemb{InSb}内部的界面作用进行驱动。
    \item 结合自由分子流模拟和第一性原理方法,提出了石墨烯在二维六方氮化硼表面快速直接生长的Cu蒸气近邻催化方式,阐明了二维六方氮化硼表面石墨烯的早期生长序列和形貌演化过程。
\end{enumerate}
\section{未来展望}
本论文对二维材料及其异质结的生长机理进行了较为系统的研究。由于二维材料及其异质结多元化的结构体系和复杂的生长机理,在通往低成本、高质量、形貌可调的二维材料及其异质结的生长的道路上仍有许多问题等待解决\chinesecolon
\begin{enumerate}[labelsep=0em,label=(\arabic*),wide]
    \item 对于石墨烯的化学气相沉积法生长,本论文探究了\cemb{Cu}衬底和气氛中的氧对于生长行为以及生长过程的作用。其他常用的催化金属衬底(如\cemb{Ni}),以及不同衬底表面处理对于石墨烯生长行为的作用仍有待研究。同时,气氛对于多层石墨烯生长模式的影响还可以进一步引入气氛中氧与氢的耦合作用,以及考虑更加复杂的流体动力学调控方法,如前驱体气流不同流速,衬底的不同迎角,反应腔不同结构等对于石墨烯生长的影响。
    \item 对于非平面极性二维材料的生长机理,在未来仍需要将生长机理研究工作扩展至更多的材料体系中,如II-VI族元素组成的极性二维材料。同时,需要研发更适合的生长模型和更高效的模拟方法,对更大尺度的极性演化进行模拟,用以探究在形成非晶态单层极性二维材料的形貌在温度涨落下演化规律,多晶形态下的晶畴、晶界运动规律等。探究更多单一极性非平面二维材料的极性调控手段。
    \item 对于二维材料异质结的生长机理,纵向二维异质结和横向二维异质结的生长机理需要得到统一的描述。需要进行更多的计算模拟探究纵向和横向二维异质结生长机理之间的相同和相异之处。只有在统一的模型描述和模拟环境中才能对不同体系,不同二维材料组合,不同生长条件下纵向和横向二维异质结的生长优先度进行判断,实现二维材料异质结的可控生长。
\end{enumerate}