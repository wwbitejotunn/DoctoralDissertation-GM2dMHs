% LTeX: language=zh-CN

\chapter{总结与展望}
\section{总结}
针对二维材料及其异质结的生长机理,本论文分别对二维材料以及二维材料组成的异质结进行了生长过程的探究以及生长机理的分析,深入讨论了衬底、气氛、生长时间、外缘催化剂、生长温度等因素对于二维材料及二维材料异质结生长过程和形貌调控的作用和影响,为二维材料及其异质结的高质量可控生长提供了理论基础和机理解释。本论文的研究分为两个大方向,分别为二维材料的生长机理和二维材料异质结的生长机理。
%//TODO 总结改写,非列表
在二维材料的生长机理中,本论文分别探究了以石墨烯为代表的平面二维材料的生长机理和以\cemb{InSb}为代表的非平面极性材料的生长机理。该方向的主要成果和创新点如下\chinesecolon
\begin{enumerate}[label=(\arabic*),wide]
    \item 化学气相沉积生长石墨烯过程衬底表面的台阶破坏了原衬底表面的对称性,为石墨烯的成核生长提供了优先生长位点和优先生长方向,使得石墨烯有机会在多晶铜的表面实现多点定向成核生长,有利于实现大面积石墨烯单晶生长的实现。生长气氛中氧的引入同时促进了石墨烯多层蚀刻和多层生长的反应。而氧气对石墨烯多层的蚀刻和穿透作用的相互竞争突破了原有\cemb{Cu}衬底上石墨烯生长的自限制作用,导致了石墨烯多层交替出现蚀刻和生长的现象,有利于实现石墨烯生长过程中的层数调控。
    \item 对于\cemb{Bi(001)}衬底上生长的单层\cemb{InSb(111)}表现出以多种混合极性平衡而成的非晶态。而当\cemb{Bi(001)}衬底上的\cemb{InSb(111)}生长至双层,第一层非晶态的\cemb{InSb}会在第二层的作用下极化至In极性。通过赝氢饱和法和四面体法,研究发现第二层\cemb{InSb}对第一层\cemb{InSb}的极化作用归功于重构\cemb{InSb}表面大幅下降的表面能。而具有低能态的表面重构结构在双层\cemb{InSb}体系中最终战胜\cemb{Bi}对于界面处\cemb{InSb}原子的吸引,将第一层\cemb{InSb}转变为\cemb{InSb}极性的构型。绘制了双层\cemb{InSb}在不同In原子化学势μIn条件下包含不同沉积原子数的生长相图。\cemb{Bi(001)}衬底上\cemb{InSb(111)}生长极性相图和极性演化机理的研究有助于加深我们对低维III-V化合物半导体纳米结构生长过程的理解,同时帮助我们对低维III-V化合物半导体纳米结构的生长过程和生长形貌进行更有效的控制。
\end{enumerate}

在二维材料异质结的生长机理中,分别探究了纵向堆叠的石墨烯/二维六方氮化硼异质结的生长机理和横向拼接的石墨烯/\cembNHS{VSe2}异质结的生长机理。该方向的主要成果和创新点如下\chinesecolon
\begin{enumerate}[label=(\arabic*),wide]
    \item 构建了一种利用\cemb{Cu}蒸气近邻催化效应在\cemb{h-BN}表面直接堆叠生长石墨烯的方法。通过结合流体力学计算和第一性原理计算,证明了\cemb{Cu}蒸气蒸气近邻催化CH4裂解的可行性探究了活性\cemb{C}原子在\cemb{h-BN}表面由单个吸附的\cemb{C}原子成核生长成\cemb{C24}团簇的生长演化序列。\cemb{C}原子团簇在\cemb{h-BN}表面的的生长形貌由初期的线形团簇(\cemb{C}原子数量$\leqslant 12$)转变为中期的环形团簇(\cemb{C}原子数量$\leqslant 17$)最后变为石墨烯结构的六边形团簇(\cemb{C}原子数量$\geqslant 19$)。利用\cemb{Cu}蒸气近邻催化效应在\cemb{h-BN}表面直接堆叠生长石墨烯的方法能够在不引入额外杂质的情况下大幅提高石墨烯在\cemb{h-BN}表面的生长速率,提升石墨烯/\cembNHS{h-BN}二维纵向异质结的产率,推进基于石墨烯/\cembNHS{h-BN}二维纵向异质结的电子器件的产业化水平。
    \item 证明对于石墨烯/\cembNHS{VSe2}横向异质结的生长,需要在生长环境中利用高温等方式进一步提升\cemb{V}原子和\cemb{Se}原子的能级水平,使其在气相中裂解成为气态的\cemb{V}自由基和\cemb{Se}自由基。发现在单层的\cemb{VSe2}具有石墨烯台阶边缘的选择性生长的特点。在热力学和动力学双重因素的驱使下,单层的\cemb{VSe2}倾向于在双层的石墨烯台阶边缘以锯齿形匹配的方式形成横向异质结。双层石墨烯台阶锯齿边缘\cemb{VSe2}的选择性生长机理有利于实现石墨烯/\cembNHS{VSe2}组成的二维横向异质结的生长位置调控,同时能够进一步提高石墨烯/\cembNHS{VSe2}组成的二维横向异质结的生长均匀性,提高异质结的合成质量。
\end{enumerate}

\section{未来展望}

%//TODO 展望