% LTeX: language=zh-CN
% TODO LIST
% 第一章 绪论
% 二维材料及其异质结
%%% 背景
%%%% 二维材料
%%%% 二维材料异质结
%%%% 实验现状
%%%% 理论现状
\chapter{绪\hspace{6pt}论}

\section{二维材料及其异质结概述}
\subsection{二维材料概述}
自从石墨烯被发现并且成功大规模制备以来\citing{RN801-2009},二维材料由于其独特的电子结构、极强的声光耦合、奇异的层间作用、多样化的性质调控手段,已经引起了大量研究者的关注。至今为止,已经有包括石墨烯在内的多种体系的二维材料被理论上预言并在成功的实验中制备,例如与石墨烯结构相似的六方氮化硼(Hexagonal boron nitride,h-BN),黑鳞(Black phosphorus,BP),二维过渡族金属硫族化合物(Transition metal dichalcogenides, TMDS)以及二维金属氧化物(Metal oxides),二维三五族半导体(III-V component semiconductors, III-Vs)等。这些二维材料来源于不同的材料体系,因此也具有各自截然不同的物理性质。%//TODO 引用

从材料性质的广度上看,多元化的二维材料赋予了二维材料器件更多,更自由的选择空间。以电子性质为例,不论是绝缘体还是半导体亦或是导体、半金属,都可以找到相对应的二维材料。
%//电子结构
在二维过渡金属硫化物中,以不同元素组成的二维金属硫化物材料虽然都具有相似的晶体结构,但其各自的电子性质确有很大的差异\citing{RN956-2015}。在二维过渡金属硫化物中,以六族元素硫,硒,碲和过渡金属铬,钼,钨组成的二维材料具有直接带隙,其禁带范围囊括0.5 eV至2.5 eV。而以其他六族元素和过渡金属组成的二维金属硫化物则表现出具有间接带隙的电子结构,且禁带范围扩展为0.9 eV至7.0 eV。如此宽泛的电子结构分布使得研究者可以仅仅使用二维过渡金属硫化物材料就可以组合出具有一型、二型和三型能带匹配的的半导体异质结,可用于制造包括发光二极管(一型能带匹配),单极、双极型电子器件(二型能带匹配),隧穿场效应管(三型能带匹配)等\citing{RN357-2016, RN958-2018}。
%//光电
多样化的的电子结构同样赋予了二维材料极宽的光响应频谱,其可以覆盖从紫外到太赫兹甚至微波的电磁波频段。当厚度逐渐减少至单原子层变为二维材料后,材料的电子性质可能会发生奇异的变化。例如在过渡族金属硫族化合物中,MoS$_2$, WS$_2$,WSe$_2$等材料的电子结构会发生变化,由块体状态下的间接带隙半导体变为二维状态下的直接带隙半导体\citing{RN984-2010,RN985-2013,RN986-2015}。电子结构的变化使得二维化的MoS$_2$的发光量子效率相比于块体状态提升了4个数量级\citing{RN984-2010}。尽管只有单层或者是几层原子的厚度,二维材料独特的电子结构使其能够与光产生奇异的相互作用。例如单层的二维二硫化钼材料可以通过激子共振的方式吸收大约10\%的对应共振频率的光线。此外,过渡族金属硫族化合物的二维化显著降低了库伦作用的介电屏蔽效应。二维过渡族金属硫化物中较强的库伦作用大幅提高了其中激子的结合能,进而提升了材料的光响应速率\citing{RN987-2014,RN988-2013}。以单层硫化钼(MoS$_2$)为例,通过理论计算预测的其激子结合能在0.5 eV到1 eV\citing{RN989-2012,RN990-2013},大大高于传统的准二维半导体量子阱\citing{RN991-1984,RN992-1981}。而对于石墨烯而言,其布里渊区内独特的狄拉克锥电子结构导致了石墨烯具有许多独特的光电性质。例如石墨烯具有约2.3\%的宽带吸收率,使其可以作为可饱和吸收器件、光子探测器以及调制器等光电器件的候选材料。同时,包括饱和吸收效应,高次谐波产生效应,合频产生效应,四波混频效应等非线性光学效应也在以石墨烯、过渡族金属硫族化合物等二维材料体系中被发现。尤其是对于过渡族金属硫族化合物体系,其面外方向的反演对称性可通过层状堆叠打破,产生在其他二维材料无法实现的偶数次谐波效应。%//FIXME 引用

%//催化
同样,二维材料独特的晶体结构和电子特性也引起了大量化学催化方面研究者的关注。得益于多样化并可精细调控的电子性质,利用二维材料制得的催化剂对那些对于低能级反应尤其有效。如氧气,一氧化碳,二氧化碳,甲烷,水等小分子之间的转化,这些反应对催化剂表面电子结构较为敏感,可以最大可能的利用二维材料表面电子结构精细可调的优势。二维材料平面化的晶体结构为活性催化基团提供了精细可调的锚定位点,催化位点的电子态可以被精细调控至相应的催化底物能级,实现整体催化活性的调控。利用这样的原理,使用石墨烯,过渡金属硫化物等二维材料作为承载面,将过渡族金属原子等活性催化单元直接嵌入二维材料的平面,可以制得新能优异的单原子催化剂\citing{RN1019-2019}。在2015年,研究者通过在石墨烯表面参杂Ni制备单原子催化剂用于酸性环境下的析氢反应,其催化性能远超镍基催化剂并且具有极高的循环稳定性\citing{RN1018-2015}。随后,通过系统的理论计算,研究者在氮参杂的石墨烯上筛选出V,Rh,Ir等金属原子作为单原子催化剂,可以显著提升析氢反应的催化性能\citing{RN1015-2020,RN1017-2019}。在2020年,有研究者报道使用ZIF8作为前驱体进行热熔解,可以获得高比表面积的Fe-N参杂石墨烯单原子催化剂,其催化性能与已经大规模商品化的Pt/C催化剂不相上下\citing{RN1016-2020}。出了石墨烯外,以过渡金属硫化物为代表的其他二维材料也作为单原子催化剂的潜在载体进入了研究者的视线。在2020年,通过使用H$_2$O$_2$对MoS$_2$进行化学腐蚀的方式,研究者在二维MoS$_2$表面获得了单S空位的催化剂,并且实现了对S空位分布的综合调控以及较强的析氢催化活性\citing{RN1020-2020}。同年,另一组研究者利用激光分子束外延的手段实现了二维CrS$_2$表面的单原子Mo参杂,CrS$_2$的表面环境极大地提升了Mo原子周围的电场水平,使得其析氢催化的性能在极大提升的同时仍然保持了较高的稳定性\citing{RN1021-2022}。
%//FIXME 引用

从材料性质的深度上看,二维材料同样拥有优异物理化学特性,具有非常高的应用潜力。在过去的十年中,二维材料所展现出的丰富的新物理激发了基础物理和技术应用方面的广泛研究。例如以石墨烯,二硫化钼($\rm{MoS_{2}}$) 以及黑鳞(BP)等为代表的二维半导体材料具有极高的载流子迁移率和相当优异的机械性能,非常适用于制造下一代低功耗高速电子器件和柔性器件。同样,以六方氮化硼,金属氟化物(如$\rm{CaF_{2}}$, $\rm{TiF_{2}}$)为代表二维绝缘体材料拥有惰性的上下表面,能够很好的与其他二维材料形成范德华界面。其作为电介质层能够极大程度地减少场效应管中沟道材料的缺陷态密度,尽可能地发挥先进沟道材料卓越的输运性能。二维材料不仅能够在现有成熟器件框架下利用其优异地物理化学性质进一步提升电子器件的性能,其独特的电子结构让研究者能够突破传统电子器件的限制,对电流以外的电子自由度进行操控。例如,石墨烯体系中超低的自旋-轨道耦合使其能够很好地保持在其上传输的自旋信息,是理想的自旋电流传输材料。而二维过渡金属硫化物体系中较强的自旋-轨道耦合性质使其能够有效地对自旋进行操控,是非常好地自旋编码材料。
这种利用电子的自旋特性,通过对电子的自旋自由度进行操作编码来进行信息处理的自旋电子器件,由于其可编程逻辑元件和非易失性信息存储器件等不同领域的应用而备受研究者的关注。随着微加工技术和大规模集成电路的发展,电子器件尺寸进一步缩小的需求与日俱增。
相比于传统电子器件中依赖电荷输运来进行信息处理的方式,通过操控电子自旋进行信息处理具有速度更快,能耗更低,集成读和稳定性更好的特点。这些特性使得具有自旋输运特性的体系在未来的数据存储,数据传输以及数据处理方面都有着极大的发展潜力[42]。%//FIXME 引用,语言
同时,二维材料独特的单层结构使得其电子中的能谷结构可以很容易地被附加的作用场调控,简并分裂的电子能谷作为额外的自由度可以看成赝自旋并用作信息处理的载体。

%//TODO 半金属 semi-metal/half-metal
此外,在二维材料中,因为具有开放的双面表面,原则上所有原子都受到表面反应的影响。同时,电荷和声子输运被严格限制在每一层中,因此其具有不同于三维体材料的特殊物理性质。以石墨烯为例,电子与蜂窝状碳晶格的相互作用使电子表现为无质量费米子,从而产生了反常的室温量子霍尔效应和极高的载流子迁移率等新的物理现象。在2008年,研究者测量到了石墨烯中电子的局域态分布,证实在石墨烯量子点中电子和空穴的由于无序效应引入的积水潭(puddle)的存在形式,解释了石墨烯中零载流子密度和非零电导同时存在的现象\citing{RN998-2008}。%//TODO。

%//超导
二维材料提供了绝佳的电子结构调控平台,使得研究者能够更好的基于现有理论发现新的物理现象、制造新的器件。而二维材料奇异的电子结构同样为新物理理论的产生提供了土壤。在2012年,研究者在单层$\rm{FeSe/SrTiO_{3}}$中发现了非经典的高温超导态,揭示了在二维材料中存在传统超导理论(BCS理论)难以解释的反常超导现象\citing{RN953-2012}。而且在$\rm{FeSe/SrTiO_{3}}$系统中,超导态仅在单层$\rm{FeSe}$的情况下被发现,并且随着$\rm{FeSe}$厚度的增长消失。更加印证了在二维材料中具有有别于块体材料的超导现象。相比于$\rm{FeSe/SrTiO_{3}}$这样复杂的系统,另一个非经典超导现象在简单的石墨烯体系中被研究者所熟知\citing{RN954-2018, RN955-2018}。在扭转双层石墨烯(Twisted bilayer graphene, TBG)中,载流子的填充水平可以简单地被双层石墨烯之间的扭转角度所调控,使其能够从莫特绝缘体转变为超导体。自2018年被发现以来,扭转双层石墨烯
%//TODO 多一些转角

%//量子计算
同时,以石墨烯为代表的多种二维材料是天然的二维电子气载体,因此在量子计算和量子晶体管方面的应用方面具有独特的优势。自2007年提出可以利用二维材料石墨烯实现自旋量子比特的理论方案\citing{RN997-2007}以来,二维材料用于量子晶体管方面的研究已有较多进展。在2008年,研究者首次在二维材料石墨烯上采用刻蚀方法实现了栅极调制的单量子点结构,成功地实现了量子晶体管\citing{RN996-2008}。2009年,研究者测量了磁场调控下石墨烯量子点的电子附加能谱(Addition Spectrum)。通过对附加能谱中电子-空穴交叉的研究,研究者得到了石墨烯量子点的线性色散和边缘限制的光谱随磁场的变化情况,为石墨烯量子点电子输运行为的磁场调控提供了理论基础\citing{RN999-2009}。在2010年,自旋态以及自旋过滤现象在石墨烯量子点中测得,研究者可以通过操控外加磁场的方式调控石墨烯量子点中的自选过滤行为\citing{RN1000-2010}。在2013年,研究者成功的在石墨烯量子点中测量了激发态能级弛豫时间,证明石墨烯量子点中能级弛豫时间和GaAs等材料基本一致,都在100纳秒量级\citing{RN1001-2013}。这些一系列的研究,使我们对石墨烯以及石墨烯量子点中的电子性质的理解更加深入,更多调控手段的引入使得以石墨烯、石墨烯量子点为基础的量子晶体管的研究得到进一步发展。在2010年,研究者在石墨烯上实现了栅极可控双量子点量子器件,研究者可以通过栅极调控量子点中的电荷数量,进而调控量子点间的电子耦合特性和激子输运特性\citing{RN1002-2010}。同年,研究者通过在石墨烯大小双量子点的方式,将大石墨烯量子点作为单电荷晶体管集成在小石墨烯量子点上,并成功测量了栅极可控的小石墨烯量子点上的电荷态\citing{RN1004-2010}。在2011年,研究者成果测量了利用单、双层石墨烯制备的平行耦合双量子点的输运性质\citing{RN1005-2011},并在2012年实现了多栅极可控的石墨烯双量子点中激发态能级的测量\citing{RN1006-2012}。同年,研究者通过在空气中使用电流烧蚀的方式制得石墨烯量子点,进一步拓宽了石墨烯量子点的大规模制备手段\citing{RN1003-2012}。其高达1.6 eV的载流子附加能(addition energy)为同时期的最高水平,可用于制备单量子晶体管。2015年,研究者探究出了石墨烯量子晶体管和超导微波腔的色散耦合方式,并实现了量子晶体管的微波场调控\citing{RN1007-2015},以及通过维波谐振器实现的两个石墨烯量子点的长程耦合\citing{RN1008-2015}。

这些尝试证明了在石墨烯上可以实现量子结构并进一步制成量子晶体管,但是由于石墨烯不是真正意义上的半导体,不具有直接带隙,因此在多场可调性和量子操作性上仍有一定局限。因此,近年来许多研究者开始探索以MoS$_2$为代表的二维过渡族金属硫族化物用于实现量子晶体管的可能性。
2015年,研究者利用纵向二维异质结作为测量平台,对MoS$_2$的输运性质进行了精确测量,获得了高达34000 cm$^2$V$^-2$s$^-1$的霍尔迁移率\citing{RN993-2015}。同年,研究者在二维WSe$_2$上的实现了门定义的单量子点结构,其隧穿势垒可由电场进行控制,而量子点的尺寸可以由相应门电极上的电压控制\citing{RN1011-2015}。相比于传统的GaAs/AlGaAs异质结,该工作进一步减小了二维过渡族金属硫化物上量子点结构的尺寸。在2016年,研究者在单层MoS$_2$上实现了单电子晶体管,并且用输运方法测量到了库伦阻塞效应\citing{RN994-2016}。2017年,研究者利用MoS$_2$和石墨烯组成的异质结实现了栅极可控的一维量子通道\citing{RN995-2017}。2017年,研究者成果测量了单层MoS$_2$在磁场中的量子输运现象及演化规律,观察到了单层MoS$_2$中量子化的电导并深入探究了MoS$_2$导带中的自选劈裂现象,并认为其可以用于制备新型二维量子自旋器件\citing{RN1009-2017}。同年,研究者在二维材料MoS2中制备了电学可调的门控双量子点结构,首次在二维材料中实现了从双量子点到单量子点的可控调节\citing{RN1014-2017}。2018年,研究者在二维MoS$_2$中实现了门控的单量子点结构,并利用量子输运方法测量到了单电子的电荷态和激子态\citing{RN1010-2018}。同时,由于二维拓朴层状材料的发现,研究者成功的在实验中观测到了二维WTe$_2$体系中由边界态传输的量子化电导,在100K地温度下,二维WTe$_2$表现出了内部绝缘,边缘导电的奇异量子态,为量子晶体管中量子态的调控提供了新的手段\citing{RN1022-2017}。
%利用二维拓扑体系中可用gate调控拓朴性的特点,在2D拓朴材料上实现Majorana费米子,在接下来设计复杂交换结构时仅仅需要利用半导体加工技术在材料上设计gate结构即可。

除了优异的材料性质之外,二维材料由于其平面化的结构特点,能够较好地与与现行集成电路使用的平面硅工艺兼容。以二维材料制成的二维纳米器件能够较为容易的在现有的芯片集成,可作为新一代纳米电子器件的候选材料。


\subsection{二维材料异质结概述}
电子器件的基本组成元件为异质结,通过对二维材料进行纵向堆叠或者平面组装而形成的二维材料异质结能够结合不同二维材料优异的物理特性,最大化的利用其性质多样化的特点\citing{RN370-2017, RN380-2012, RN369-2014, RN371-2014, RN357-2016, RN353-2017, RN385-2014, RN351-2014, RN316-2018, RN387-2012, RN368-2017}。 而二维材料较弱的层间作用和相似的平面结构为二维材料异质结稳定界面的形成提供了保证。得益于其优秀的性质,二维材料异质结近年来发展迅速。早在2010年, Dean等人通过将剥离的单层石墨烯转移到六方氮化硼上,制成了由两个单原子层的二维材料组成的异质结。在那之后,由于其独特的界面激子效应和优异的光伏相应特性,包括石墨烯/六方氮化硼,石墨烯/过渡金属硫化物,过渡金属硫化物/过渡金属硫化物在内的多种二维材料异质结引起了大量研究者的关注。迄今为止,已经有多种二维异质结通过实验合成\citing{RN961-2018, RN960-2018, RN351-2014}并用于制造场效应管\citing{RN386-2012, RN387-2012}、存储器件\citing{RN389-2013, RN388-2011}、光电器件\citing{RN390-2013}等。二维异质结使得功能更强大、特性更新颖的电子器件的出现成为了可能。

%//二维材料异质结
由于机械剥离和转移技术的成熟,将二维材料进行纵向堆叠而成的纵向异质结首先进入研究者的视线。2010 年,研究者通过机械剥离的方式的将单层石墨烯转移到二维六方氮化硼上,形成了具有原子层级厚度的异质结\citing{RN959-2010}。从那时起,包括 graphene/h-BN,graphene/TMDs,TMDs/TMDs 在内的大量纵向二维异质结引起了研究者的广泛关注\citing{RN309-2015, RN384-2015, RN319-2017, RN383-2012, RN368-2017},包括界面激子效应,谷电子效应和光伏响应等优异的物理性质也被逐一从二维材料异质结之中发掘出来。例如,自2015年起,已有研究者通过理论计算发现,可以将单层SnSe$_2$与单层MoS$_2$进行堆叠形成一型能带匹配的半导体异质结。其具有的超快电子-空穴复合效应使其非常适合用于制造发光二极管等光电器件\citing{RN966-2019,RN978-2017,RN979-2015}。对于二型能带匹配,由于导带底和价带顶分属于不同材料中,因此二型能带匹配可以用于电子和空穴的高效分离。若是将二维材料进行堆叠形成具有二型能带匹配的异质结,那么二维材料之间的范德华区域将对电子和空穴的相互作用产生屏蔽效应,可以对电子和空穴进行有效分离,导致其激子寿命长于普通材料\citing{RN969-2015,RN970-2015}。自2018年起,已有研究者分别从实验和理论计算的角度证实,在二维过渡族金属硫族化合物异质结,二维III-VI族化合物异质结等二维异质结体系中均可观测到持续时间非常长的电子-空穴分离态\citing{RN976-2020,RN975-2019,RN972-2017,RN977-2018}。而在二维Janus材料和二维钼硫硒化合物(MoSSe)组成的具有二型能带匹配的双原子层异质结中,在2019年已有理论计算预测其具有长达16.5 ns的激子寿命\citing{RN971-2019}。对于具有三型能带匹配的异质结,其破缺的能带结构允许载流子在不同材料的能带之间隧穿,使其能够作为隧道场效应管的基础结构。例如,黑磷/硫化锌(Phosphorene/SnS$_2$),黑磷/硒化锌(Phosphorene/SnSe$_2$),硒化钨/硒化锌(WSe$_2$/SnSe$_2$)等材料可以组成具有三型能带匹配的异质结,已经有工作通过理论计算预测可以在这类二维异质结中观察到载流子的带间输运现象,是实现二维隧道场效应管的候选材料\citing{RN981-2018,RN982-2016,RN983-2017}。同时,由于三型能带匹配中电子和空穴的传输速度远高于二型能带匹配,大量的电子和空穴可以被迅速得分离到一直接种不同的材料上,使异质结内部产生一个较强的内建电场,并且显示出类似于半金属的特性。这样的特点使得具有三型能带匹配的异质结非常适合用于制造下一代新型热光伏太阳能电池\citing{RN980-2019}。
不仅如此,二维材料之间的能带匹配还能够通过外加电场等方式进行改变。例如,可以在单层硒化锌(SnSe$_2$)与单层硫化钼(MoS$_2$)形成的异质结中附加外加电场,使其从原来的一型能带匹配变为二型能带匹配\citing{RN966-2019}。在单层硫化锗(GeS)和砷烯(Arsenene)形成的异质结体系中,外加正电场可以其保持原有的二型能带匹配,而外加负电场可以使其转变为一型能带匹配\citing{RN967-2019}。而在磷/硫化锌(Phosphorene/SnS$_2$)形成的具有三型能带匹配的异质结中,外加负电场可以实现一型,二型,三型的能带匹配转变\citing{RN981-2018}。除了外加电场外,最新的研究发现可以通过施加应变应力的方式,将单层硼磷化合物(Boron phosphide)和单层MoSSe形成的的一型能带匹配转变为二型能带匹配\citing{RN968-2021}。

作为二维异质结的组成材料,引起研究者关注的不仅仅是那些具有半导体特性的二维材料。作为二维材料中的绝缘体,六方氮化硼在具有较大的带隙的同时也保有二维材料高质量面内结构的特点。趋近完美的二维晶体结构使得六方氮化硼在禁带内具有极低的缺陷态密度以及较高的击穿电压,使其能够在隧穿器件中作为非常好的势垒材料。同时,利用六方氮化硼高耐压的特性,将石墨烯和六方氮化硼进行堆叠,能够制成具有超薄介电层的电容器。超薄的介电层能够将小至单位电荷的变化情况传导到电极之中,使之成为量子电容器。量子电容器可用于测量量子输运体系中极其微小的电荷转移情况。运用类似的思路,将石墨烯、过渡金属硫化物以及黑磷等材料纵向堆叠成类三明治结构所制得的量子电容器也被大量研究者所关注。而最早发现的二维材料,石墨烯,其独特的电子结构使得可以通过引入外置栅极的方式对费米能级和态密度的位置进行调控,以此来操纵穿过势垒的隧穿电流的大小,适合用作隧穿器件中的源极和漏极材料。

与传统材料界面处的共价键相反,二维材料异质结构中各层之间相对较弱的范德华力极大地放宽了晶格匹配和化合价的要求,最终实现了更宽的异质结构相空间。尽管很弱,但范德华相互作用仍然主导了跨界面的各种类型的耦合。例如,电荷在界面处重新分布以平衡化学势,这会影响诸如电荷屏蔽、能带弯曲和载流子耗尽等现象。尽管传统的三维异质结构和二维范德华结构均发生界面电荷转移,但二者存在显著差异。首先,范德华界面处较小的电子交叠将电荷层间转移的过程限制为相对低效的隧穿和跳跃。其次,由于范德华界面处介电屏蔽的减少而导致激子结合能的增加,消除了传统异质结中决定电荷转移过程的能带偏移。这些因素都导致在传统异质结的理论框架下,二维材料范德华异质结中不同材料直接的电荷转移是缓慢的。但是,最近的研究发现了在部分二维材料组成的范德华异质结构中,存在超快的电荷转移过程,包括MoS$_2$/WS$_2$,MoS$_2$/石墨烯,WS$_2$/石墨烯量子点和MoS$_2$/有机分子异质结构等\citing{RN523-2017, RN308-2014, RN520-2016}。%//FIXME 检查一下这几个例子
在二维超导体中,界面耦合也会强烈影响超导临界温度。例如,FeSe是研究最多的二维超导体之一,其体超导临界温度(Tc)约为≈8K。然而,通过在SrTiO3上生长单层FeSe,由于界面增强的电子-声子耦合,Tc显着提高到≈100K\citing{RN747-2014, RN746-2013, RN748-2018}。
除电子耦合外,自旋自由度还引入了界面磁耦合。自旋之间的交换相互作用决定了相邻电子之间的自旋排列方式,平行或反平行,对应于铁磁和反铁磁排序。对于铁磁层和反铁磁层形成的界面,在外部磁场为零的情况下下,界面处的排序类型由交换相互作用决定。通过施加外部磁场,反铁磁性层的磁化方向可以翻转到与外部磁场平行对其。这是因为即使交换相互作用使得反铁磁性层的界面能增加,但是外部磁场平行对齐的电子自旋仍然可以使总能量下降。通过交换相互作用和超快速电荷转移,当单层WSe2与CrI3集成在垂直异质结构中时,可以观察到其山谷极化得到了明显改善\citing{RN751-2018, RN752-2019}。//FIXME 修改

\section{二维材料及其异质结的生长制备和机理研究}
二维材料以其优异的电学、光学、化学、机械特性获得了大量研究者的关注。这些新颖的特性同时也为二维材料及其异质结带来了巨大的应用潜力和极高的理论研究价值。为了最大程度的研究了利用二维材料的诸多优异特性,研究者们一直在致力于研究二维材料及其异质结的高质量,高可控,大规模,低成本的合成方法。一般认为,二维材料及其异质结的合成方法可以分为两类,分别是自顶向下(top-down)合成方法和自底向上(bottom-up)合成方法。

%参考用
%二维材料通常通过三种不同的技术获得——微机械解理、物理气相沉积(PVD)和化学气相沉积(CVD)。微机械解理(即所谓的透明胶带法)涉及从大块晶体中剥离单层和几层薄片。该方法提供了适合微米级电子和光学设备的晶体样品。最近在去角质技术方面取得的进展使研究人员能够有选择地分离、成型和堆叠厚度和方向可控的薄片6。然而,微观机械解理是劳动密集型的,并且仅限于具有层状块状同素异形体的材料(例如,石墨剥落成石墨烯)。该方法的可扩展性通过基于溶液的剥离技术170171得到了显著增强,该技术以牺牲薄片尺寸为代价生成大量悬浮2D薄片。PVD使用加热的原子源在基板12上可控地沉积材料。对于合成元素2D材料(SE2DM)的生长,PVD通常需要超高真空(UHV)条件和校准良好的高纯度原子源。在超高压条件下制备的样品可以通过表面敏感技术进行现场研究,以保持材料的原始状态。CVD利用气体、液体或固体前体在受控气氛下的分解或反应来生长2D材料172173。对于CVD,催化活性底物通常是最有效的,并且可以在大气压到UHV的压力范围内进行合成。生长机制因基板而异:前体不溶性低的基板可能催化原子薄膜的自限生长,而其他基板则溶解大量前体,冷却后分离到表面。表面分解或分离利用生长衬底上的固态前体。当衬底被加热到足以激活升华或扩散过程时,生长发生,产生富含前体元素的表面层,然后冷凝为2D相。衬底选择仅限于包含前体元素的衬底,例如在SiC174上合成外延石墨烯。此外,可以从前体衬底上的非均匀薄膜(10–100nm厚度)开始,其加热有助于基于扩散的分离过程85175。
\subsection{非极性二维材料}
%//TODO 石墨烯生长
%//TODO CVD
%//TODO 自限制生长

%%非极性二维材料
\subsection{极性二维材料}
%%极性二维材料
%%III-V半导体生长
\subsection{生长工艺及机理模型}
%//TODO https://sci-hub.se/10.1038/s41570-016-0014

%%纵向二维异质结
%%横向二维异质结
\subsection{纵向堆叠二维材料异质结}
对于纵向二维异质结,除了机械剥离法外,在已经制备的二维异质结中,现有的制备方法有图形再生长
(patterned regrowth)[45]、刻蚀再生长(etching regrowth)[46]、化学转化法(chemical 
conversion methods)[47]、一步法化学气相沉积(一步法, one-step CVD methods) [48,49]
和两步法化学气相沉积(两步法, two-step CVD method) [43,50]等。

\subsection{横向堆叠二维材料异质结}
横向二维异质结独特的结构使之具有许多出色的性能,在应用方面具有广阔的
前景。它们的电子特性也得到了广泛研究[29,38]。利用 MoS2-NbS2 横向异质结中的交错禁
带和弱耦合状态产生的传输间隙,可以制造出开关电流比为 106~107,漏电流约为 108
uA 的场效应晶体管[25];利用其纳米尺寸和多组分光学性质,横向 TMD 异质结可以用作单分
子探测以及制造具有可调光响应的纳米器件[39];基于 WSe2-WS2 横向异质结的 p-n 结二极
管和光电二极管,表现出了良好的整流特性并能产生较大的光电流[40];这些独特的优异
性能使得横向二位异质结为制造更高性能互补逻辑电路、高频器件、及光电探测器等电子、
光电子器件带来了新的可能。同时,由于横向二维异质结中普遍存在的自旋过滤效应,进
一步推动了自旋电子学的发展。例如 ZGNR/g-C3N4 异质结和 h-BN/zigzag graphene 构建的
异质结,都有着极高的自旋过滤效率[41]。

\subsection{生长工艺及机理模型}


对于横向二维异质结,最为常用的合成方法是 CVD 法。横向的石墨烯和 hBN 异质结,
最先在 2010 年被 Feng Liu, Pulickel M. Ajayan 等人采用一步法合成出来。这个单层结构的
异质结在空间上是随机分布的,并且其带隙不同于纯石墨烯和 hBN[51]。在 2012 年,复合
石墨烯和六方氮化硼薄层通过两步法合成[52],但是其形状和边缘难以控制。A. T. Charlie 
Johnson 等人采用了常压 CVD 法合成了有平直界面,且六方氮化硼沿着石墨烯晶体取向生
长的二维异质结,其界面的清晰程度可以通过控制生长条件进行调控[53]。An-Ping Li,Gong 
Gu 等人采用两步法合成了石墨烯和六方氮化硼的异质结,通过对于第一步中合成的石墨
烯层边缘的进行氢原子刻蚀,实现了 zigzag 形态的界面[54]。
随着更多二维材料的涌现,过渡金属硫化物,特别是 MoS2 和 WSe2 受到了广泛关注。
2014 年,Xiangfeng, Duan 等人通过横向外延的方式,用一步法制备了具有渐变无隙界面的
WS2/WSe2 和 MoS2/MoSe2 异质结并且得到了很好的电子和光电特性[55]。Wu Zhou 和
Pulickel M. Ajayan 等人采用一步法同时合成了横向和纵向异质结[3]。除了 CVD 方法外,
MoSe2/MoS2 异质结也通过电子束光刻的方式合成[56]。
由于一步合成的方法基于材料的自组装,我们很难控制二维异质结的形状和尺寸。同
时对于 WX2 体系,因为金属和硫族元素同时发生了变化,一步法很难生长如同 WSe2-MoS2
这样的异质结结,两步法的提出克服了这些困难。2015 年,有原子级的清晰界面的
WSe2/MoS2 异质结通过两步外延生长方法成功合成[57]。 同时,Gong 等人设计了一个两
步生成纵向和横向 WSe2/MoSe2 异质结的方法,能产生 169um 的异质结,且交叉污染相比
之前的一步法要小很多。Cai 等人设计了一种新颖的采用离子交换方式的两步生长方法。
采用这种方法,MoS2/ WSe2 异质结尺寸能长到 100um[58]。
厚度调制的二维异质结也能采用 CVD 方法进行制备。Zhang 等人,制备了单层和双层
阶梯状的异质结[59]。He 等人采用了 CVD 方法得到了了不同层数的 MoSe2 结。这对于层
数控制的大尺寸异质结生长具有参考价值[60]。
同时,图案化再生长和化学转化法也被用于合成二维异质结。在 2012 年,Mark P 等人
使用了两步法合成了具有空间调制效应的六方氮化硼/石墨烯的异质结。他们先用 CVD 方
法合成了石墨烯层,然后将石墨烯刻蚀形成特定图案,然后实现了 h-BN 层的空间选择性
的生长[45]。Liu 等人,采用了两步 CVD 法合成了一个具有可控形状、清晰界面、大尺寸
的六方氮化硼/石墨烯的异质结[61]。同时,MoSe2/MoS2 异质结也可以通过电子束光刻的方
式合成。这种方法能够提供清晰界面,并且能够对图案和生长过程进行控制[56]。
二维异质结也能使用物理气相转移,机械方法或者光刻等方法进行制备。Huang 等人
采用气-固生长方法合成了一个无缝的高质量的横向 MoSe2–WSe2 异质结。并具有很好的
光致发光特性[48]。Tosun 等人通过机械法合成了一个单层和多层的一型异质结。这种通过
厚度调制的二维异质结设计过程非常简单,不用涉及复杂的化学过程[62]。Jamilpanah 等人
采用了双极电极沉积技术合成了一型和二型异质结。这个方法可以在室温下进行,并且成
本低廉[49]。在这些生长方法里,图形再生长法在制备上显得较为繁琐;刻蚀再生长、化学
转化法则是利用化学反应对第一层单层二维材料薄膜进行刻蚀再生长或者直接转化,由于
化学反应的复杂性,使得我们不能很好对二维异质结的生长进行较为精细的控制。而一步
法虽然简单易操作,但是目前使用该方法制备出的二维材料异质结都在几十微米范围以内
[3,63] ,并且使用一步法制备的二维异质结在两种二维材料交接处容易产生交叉污染或形
成合金相,较难得到清晰的界面 [3,50] 。


对于纵向二维异质结,由于是层叠结构,异质结的生长与三维薄膜生长较为相似,层
与层之间的作用力由化学键变为范德瓦尔兹力,我们可以利用三维薄膜生长模型进行大致
地描述。考虑被沉积物自由能(γad),衬底表面自由能(γsub)以及界面自由能(γinter)
之间的竞争作用。当沉积物与衬底之间有很强的作用力时,γad + γinter < 𝛾𝑠𝑢𝑏,表现出层
状生长;当被沉积物与衬底表面之间的相互作用较弱的时候,被沉积原子的能量和界面的
能量之和大于或近似等于衬底表面的自由能,也就是γad + γinter ≥ 𝛾𝑠𝑢𝑏,此时表现出岛状
生长或者是层状-岛状生长的模式。对于纵向二维异质结,较弱的层间作用使其在生长过程
中一般表现出岛状(VW)的生长模式,但在掺杂和缺陷存在的情况下可能会转变为层状
(FM)或者层状-岛状(SK)的生长模式。[64]
\section{多尺度生长机理}%option

\section{本论文研究的主要思路及内容}