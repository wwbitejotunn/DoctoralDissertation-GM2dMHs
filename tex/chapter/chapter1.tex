% LTeX: language=zh-CN
% TODO LIST
% 第一章 绪论
% 二维材料及其异质结
%%% 背景
%%%% 二维材料
%%%% 二维材料异质结
%%%% 实验现状
%%%% 理论现状
\chapter{绪\hspace{6pt}论}

\section{二维材料及其异质结概述}
自从石墨烯被发现并且成功大规模制备以来\citing{RN801-2009},二维材料由于其独特的电子结构、极强的声光耦合、奇异的层间作用、多样化的性质调控手段,已经引起了大量研究者的关注。至今为止,已经有包括石墨烯在内的多种体系的二维材料被理论上预言并在成功的实验中制备,例如与石墨烯结构相似的六方氮化硼(Hexagonal boron nitride,h-BN),黑鳞(Black phosphorus,BP),过渡族金属硫化物(Transition metal dichalcogenides, TMDS)以及二维金属氧化物(Metal oxides),二维三五族半导体(III-V component semiconductors, III-Vs)等。这些二维材料来源于不同的材料体系,因此也具有各自截然不同的物理性质。多样化的材料体系使得二维材料的性质具有广阔的潜力。以电子性质为例,不论是绝缘体还是半导体亦或是导体、半金属,都可以找到相对应的二维材料。

从材料性质的广度上看,多元化的二维材料赋予了二维材料器件更多,更自由的选择空间。
%//光电
多样化的的电子结构赋予了二维材料极宽的光响应频谱,其可以覆盖从紫外光到太赫兹乃至于微波的频段。尽管只有单层或者是几层原子的厚度,二维材料独特的电子结构使其能够与光产生奇异的相互作用。例如单层的二维二硫化钼材料,通过激子共振的方式,可以吸收大约10\%对应共振频率的光线。而对于石墨烯而言,其布里渊区内独特的狄拉克锥电子结构导致了石墨烯具有许多独特的光电性质。例如石墨烯具有约2.3\%的宽带吸收率,使其可以作为可饱和吸收其,光子探测器以及调制器等光电器件的候选材料。同时,包括饱和吸收效应,高次谐波产生效应,合频产生效应,四波混频效应等非线性光学效应也在以石墨烯、过渡族金属硫化物等二维材料体系中被发现。尤其是对于过渡族金属硫化物体系,其面外方向的反演对称性可通过层状堆叠打破,使其能够产生在其他二维材料无法实现的二次谐波效应。
%//催化
同样,二维材料独特的晶体结构和电子特性也引起了大量化学催化方面研究者的关注。得益于多样化并可精细调控的电子性质,利用二维材料制得的催化剂对那些对于低能级反应尤其有效。如氧气,一氧化碳,二氧化碳,甲烷,水等小分子之间的转化,这些反应对催化剂表面电子结构较为敏感,可以最大可能的利用二维材料表面电子结构精细可调的优势。二维材料平面化的晶体结构为活性催化基团提供了精细可调的锚定位点,催化位点的电子态可以被精细调控至相应的催化底物能级,实现整体催化活性的调控。利用这样的原理,使用石墨烯,过渡金属硫化物等二维材料作为承载面,将过渡族金属等活性催化单元直接嵌入二维材料的平面,可以制得新能优异的单原子催化剂。

从材料性质的深度上看,二维材料同样拥有优异物理化学特性,具有非常高的应用潜力。例如以石墨烯,二硫化钼($ \rm{MoS_{2}} $) 以及黑鳞(BP)等为代表的二维半导体材料具有极高的载流子迁移率和相当优异的机械性能,非常适用于制造下一代低功耗高速电子器件和柔性器件。同样,以六方氮化硼,金属氟化物(如 $ \rm{CaF_{2}} $ , $\rm{TiF_{2}}$ )为代表二维绝缘体材料拥有惰性的上下表面,能够很好的与其他二维材料形成范德华界面。其作为电介质层能够极大程度地减少场效应管中沟道材料的缺陷态密度,尽可能地发挥先进沟道材料卓越的输运性能。二维材料不仅能够在现有成熟器件框架下利用其优异地物理化学性质进一步提升电子器件的性能,其独特的电子结构让研究者能够突破传统电子器件的限制,对电流以外的电子自由度进行操控。例如,石墨烯体系中超低的自旋-轨道耦合使其能够很好地保持在其上传输的自旋信息,是理想的自旋电流传输材料。二维过渡金属硫化物体系中较强的自旋-轨道耦合性质使其能够有效地对自旋进行操控,是非常好地自旋编码材料。同时,二维材料独特的单层结构使得其电子中的能谷结构可以很容易地被附加的作用场调控,简并分裂的电子能谷作为额外的自由度可以看成赝自旋并用作信息处理的载体。
%//超导
二维材料提供了绝佳的电子结构调控平台,使得研究者能够更好的基于现有理论发现新的物理现象、制造新的器件。而二维材料奇异的电子结构同样为新物理理论的产生提供了土壤。在2012年,薛其坤教授的研究团队在单层$\rm{FeSe/SrTiO_{3}}$中发现了非经典的高温超导态,揭示了在二维材料中存在传统超导理论(BCS理论)难以解释的反常超导现象\citing{RN953-2012}。而且在$\rm{FeSe/SrTiO_{3}}$系统中,超导态仅在单层$\rm{FeSe}$的情况下被发现,并且随着$\rm{FeSe}$厚度的增长消失。更加印证了在二维材料中具有有别于块体材料的超导现象。相比于$\rm{FeSe/SrTiO_{3}}$这样复杂的系统,另一个非经典超导现象在简单的石墨烯体系中被研究者所熟知\citing{RN954-2018, RN955-2018}。在扭转双层石墨烯(twisted bilayer graphene, TBG)中,载流子的填充水平可以简单地被双层石墨烯之间的扭转角度所调控,使其能够从莫特绝缘体转变为超导体。

除了优异的材料性质之外,二维材料由于其平面化的结构特点,能够较好地与与现行集成电路使用的平面硅工艺兼容。以二维材料制成的二维纳米器件能够较为容易的在现有的芯片集成,可作为新一代纳米电子器件的候选材料。

电子器件的基本组成元件为异质结,通过对二维材料进行纵向堆叠或者平面组装而形成的二维材料异质结能够结合不同二维材料优异的物理特性,最大化的利用其性质多样化的特点\citing{RN370-2017, RN380-2012, RN369-2014, RN371-2014, RN357-2016, RN353-2017, RN385-2014, RN351-2014, RN316-2018, RN387-2012, RN368-2017}。 而二维材料较弱的层间作用和相似的平面结构为二维材料异质结稳定界面的形成提供了保证。得益于其优秀的性质,二维材料异质结近年来发展迅速。早在2010年, Dean等人通过将剥离的单层石墨烯转移到六方氮化硼上,制成了由两个单原子层的二维材料组成的异质结。在那之后,由于其独特的界面激子效应和优异的光伏相应特性,包括石墨烯/六方氮化硼,石墨烯/过渡金属硫化物,过渡金属硫化物/过渡金属硫化物在内的多种二维材料异质结引起了大量研究者的关注。

%//TODO 二维材料异质结
由于机械剥离和转移技术的成熟,将二维材料进行纵向堆叠而成的纵向异质结首先进入研究者的视线。六方氮化硼作为二维材料中的绝缘体,在具有较大的带隙的同时也保有二维材料高质量面内结构的特点。趋近完美的二维晶体结构导致了六方氮化硼在禁带内具有极低的缺陷态密度以及较高的击穿电压,使其能够在隧穿器件中作为非常好的势垒材料。而最早发现的二维材料,石墨烯,其独特的电子结构使得可以通过引入外置栅极的方式对费米能级和态密度的位置进行调控,以此来操纵穿过势垒的隧穿电流的大小,适合用作隧穿器件中的源极和漏极材料。
%//TODO 隧穿器件
不仅如此,利用六方氮化硼高耐压的特性,将石墨烯和六方氮化硼进行堆叠,能够制成具有超薄介电层的电容器。超薄的介电层能够将小至单位电荷的变化情况传导到电极之中,使之成为量子电容器。量子电容器可用于测量量子输运体系中极其微小的电荷转移情况。运用类似的思路,将石墨烯、过渡金属硫化物以及黑磷等材料纵向堆叠成类三明治结构所制得的量子电容器也被大量研究者所关注。
%//TODO 光电器件

%//TODO 催化器件?


\section{二维材料的生长制备和机理研究}
\subsection{非极性二维材料}
%%非极性二维材料
\subsection{极性二维材料}
%%极性二维材料
\subsection{生长工艺及机理模型}

\section{二维材料异质结的生长制备和机理研究}
%%纵向二维异质结
%%横向二维异质结
\subsection{纵向堆叠二维材料异质结}
\subsection{横向堆叠二维材料异质结}
\subsection{生长工艺及机理模型}

\section{本论文研究的主要思路及内容}