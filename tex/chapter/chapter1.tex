% LTeX: language=zh-CN
% TODO LIST
% 第一章 绪论
% 二维材料及其异质结
%%% 背景
%%%% 二维材料
%%%% 二维材料异质结
%%%% 实验现状
%%%% 理论现状
\chapter{绪\hspace{6pt}论}

\section{二维材料及其异质结}
自从石墨烯被发现以来,二维材料由于其独特的电子结构、极强的声光耦合、奇异的层间作用、多样化的性质调控手段,已经引起了大量研究者的关注。至今为止,已经由包括石墨烯在内的多种体系的二维材料被理论上预言并在成功的实验中制备,其中包括与石墨烯结构相似的六方氮化硼(h-BN),黑鳞(Black phosphorus),过渡族金属硫化物(Transition metal dichalcogenides, TMDS)以及二维三五族半导体(III-V component semiconductors)等。这些二维材料来源于不同的材料体系,因此也具有各自截然不同的物理性质。多样化的材料体系使得二维材料的性质具有宽广的可能性。以电子性质为例,不论是绝缘体还是半导体亦或是导体、半金属,都可以找到相对应的二维材料。从材料性质的广度上看,二维材料多样化的特性赋予了二维材料器件更多,更自由的选择空间。从材料性质的深度上看,二维材料同样拥有优异物理化学特性,具有非常高的应用潜力。例如以
%//TODO
为代表的二维半导体材料具有极高的载流子迁移率和相当优异的机械性能,非常适用于制造下一代低功耗高速电子器件和柔性器件。
%//TODO more example need.
出了优异的材料性质之外,二维材料由于其平面化的特点,能够较好地与与现行集成电路使用的平面硅工艺兼容。以二维材料制成的二维纳米器件能够较为容易的在现有芯片上地集成,可作为新一代纳米电子器件的候选材料。


\section{二维材料及其异质结的生长制备}
近年来,二维材料及二维材料异质结的研究发展迅速。

\section{二维材料及其异质结的生长机理}

