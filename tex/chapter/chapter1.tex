% LTeX: language=zh-CN
% TODO LIST
% 第一章 绪论
% 二维材料及其异质结
%%% 背景
%%%% 二维材料
%%%% 二维材料异质结
%%%% 实验现状
%%%% 理论现状
\chapter{绪\hspace{6pt}论}

\section{二维材料及其异质结概述}
自从石墨烯被发现并且成功大规模制备以来\citing{RN801-2009},二维材料由于其独特的电子结构、极强的声光耦合、奇异的层间作用、多样化的性质调控手段,已经引起了大量研究者的关注。至今为止,已经有包括石墨烯在内的多种体系的二维材料被理论上预言并在成功的实验中制备,例如与石墨烯结构相似的六方氮化硼(Hexagonal boron nitride,h-BN),黑鳞(Black phosphorus,BP),二维过渡族金属硫化物(Transition metal dichalcogenides, TMDS)以及二维金属氧化物(Metal oxides),二维三五族半导体(III-V component semiconductors, III-Vs)等。这些二维材料来源于不同的材料体系,因此也具有各自截然不同的物理性质。多样化的材料体系使得二维材料的性质具有广阔的潜力。以电子性质为例,不论是绝缘体还是半导体亦或是导体、半金属,都可以找到相对应的二维材料。

从材料性质的广度上看,多元化的二维材料赋予了二维材料器件更多,更自由的选择空间。
%//电子结构
以二维过渡金属硫化物为例,以不同元素组成的二维金属硫化物材料虽然都具有相似的晶体结构,但其各自的电子性质确有很大的差异\citing{RN956-2015}。在二维过渡金属硫化物中,以六族元素硫,硒,碲和过渡金属铬,钼,钨组成的二维材料具有直接带隙,其禁带范围囊括0.5eV至2.5eV。而以其他六族元素和过渡金属组成的二维金属硫化物则表现出具有间接带隙的电子结构,且禁带范围扩展为0.9eV至7.0eV。如此宽泛的电子结构分布使得研究者可以仅仅使用二维过渡金属硫化物材料就可以组合出具有一型、二型和三型能带匹配的的半导体异质结,可用于制造包括发光二极管(一型能带匹配),单极、双极型电子器件(二型能带匹配),隧穿场效应管(三型能带匹配)等\citing{RN357-2016, RN958-2018}。

%//光电
多样化的的电子结构同样赋予了二维材料极宽的光响应频谱,其可以覆盖从紫外光到太赫兹乃至于微波的频段。尽管只有单层或者是几层原子的厚度,二维材料独特的电子结构使其能够与光产生奇异的相互作用。例如单层的二维二硫化钼材料,通过激子共振的方式,可以吸收大约10\%对应共振频率的光线。而对于石墨烯而言,其布里渊区内独特的狄拉克锥电子结构导致了石墨烯具有许多独特的光电性质。例如石墨烯具有约2.3\%的宽带吸收率,使其可以作为可饱和吸收其,光子探测器以及调制器等光电器件的候选材料。同时,包括饱和吸收效应,高次谐波产生效应,合频产生效应,四波混频效应等非线性光学效应也在以石墨烯、过渡族金属硫化物等二维材料体系中被发现。尤其是对于过渡族金属硫化物体系,其面外方向的反演对称性可通过层状堆叠打破,产生在其他二维材料无法实现的偶数次谐波效应。
%//TODO https://sci-hub.se/https://link.springer.com/article/10.1007/s12274-020-3126-9
%//TODO 可调带隙,直接-间接带隙转换
%//TODO 激子结合能,
%//TODO https://sci-hub.se/https://onlinelibrary.wiley.com/doi/abs/10.1002/aelm.202001174
%//TODO 不同能带匹配对应的不同的光电器件


%//催化
同样,二维材料独特的晶体结构和电子特性也引起了大量化学催化方面研究者的关注。得益于多样化并可精细调控的电子性质,利用二维材料制得的催化剂对那些对于低能级反应尤其有效。如氧气,一氧化碳,二氧化碳,甲烷,水等小分子之间的转化,这些反应对催化剂表面电子结构较为敏感,可以最大可能的利用二维材料表面电子结构精细可调的优势。二维材料平面化的晶体结构为活性催化基团提供了精细可调的锚定位点,催化位点的电子态可以被精细调控至相应的催化底物能级,实现整体催化活性的调控。利用这样的原理,使用石墨烯,过渡金属硫化物等二维材料作为承载面,将过渡族金属原子等活性催化单元直接嵌入二维材料的平面,可以制得新能优异的单原子催化剂。

从材料性质的深度上看,二维材料同样拥有优异物理化学特性,具有非常高的应用潜力。例如以石墨烯,二硫化钼($\rm{MoS_{2}}$) 以及黑鳞(BP)等为代表的二维半导体材料具有极高的载流子迁移率和相当优异的机械性能,非常适用于制造下一代低功耗高速电子器件和柔性器件。同样,以六方氮化硼,金属氟化物(如 $\rm{CaF_{2}}$ , $\rm{TiF_{2}}$ )为代表二维绝缘体材料拥有惰性的上下表面,能够很好的与其他二维材料形成范德华界面。其作为电介质层能够极大程度地减少场效应管中沟道材料的缺陷态密度,尽可能地发挥先进沟道材料卓越的输运性能。二维材料不仅能够在现有成熟器件框架下利用其优异地物理化学性质进一步提升电子器件的性能,其独特的电子结构让研究者能够突破传统电子器件的限制,对电流以外的电子自由度进行操控。例如,石墨烯体系中超低的自旋-轨道耦合使其能够很好地保持在其上传输的自旋信息,是理想的自旋电流传输材料。而二维过渡金属硫化物体系中较强的自旋-轨道耦合性质使其能够有效地对自旋进行操控,是非常好地自旋编码材料。同时,二维材料独特的单层结构使得其电子中的能谷结构可以很容易地被附加的作用场调控,简并分裂的电子能谷作为额外的自由度可以看成赝自旋并用作信息处理的载体。
%//TODO 半金属 semi-metal/half-metal
%//超导
二维材料提供了绝佳的电子结构调控平台,使得研究者能够更好的基于现有理论发现新的物理现象、制造新的器件。而二维材料奇异的电子结构同样为新物理理论的产生提供了土壤。在2012年,薛其坤教授的研究团队在单层$\rm{FeSe/SrTiO_{3}}$中发现了非经典的高温超导态,揭示了在二维材料中存在传统超导理论(BCS理论)难以解释的反常超导现象\citing{RN953-2012}。而且在$\rm{FeSe/SrTiO_{3}}$系统中,超导态仅在单层$\rm{FeSe}$的情况下被发现,并且随着$\rm{FeSe}$厚度的增长消失。更加印证了在二维材料中具有有别于块体材料的超导现象。相比于$\rm{FeSe/SrTiO_{3}}$这样复杂的系统,另一个非经典超导现象在简单的石墨烯体系中被研究者所熟知\citing{RN954-2018, RN955-2018}。在扭转双层石墨烯(twisted bilayer graphene, TBG)中,载流子的填充水平可以简单地被双层石墨烯之间的扭转角度所调控,使其能够从莫特绝缘体转变为超导体。
%//TODO 量子计算


除了优异的材料性质之外,二维材料由于其平面化的结构特点,能够较好地与与现行集成电路使用的平面硅工艺兼容。以二维材料制成的二维纳米器件能够较为容易的在现有的芯片集成,可作为新一代纳米电子器件的候选材料。

电子器件的基本组成元件为异质结,通过对二维材料进行纵向堆叠或者平面组装而形成的二维材料异质结能够结合不同二维材料优异的物理特性,最大化的利用其性质多样化的特点\citing{RN370-2017, RN380-2012, RN369-2014, RN371-2014, RN357-2016, RN353-2017, RN385-2014, RN351-2014, RN316-2018, RN387-2012, RN368-2017}。 而二维材料较弱的层间作用和相似的平面结构为二维材料异质结稳定界面的形成提供了保证。得益于其优秀的性质,二维材料异质结近年来发展迅速。早在2010年, Dean等人通过将剥离的单层石墨烯转移到六方氮化硼上,制成了由两个单原子层的二维材料组成的异质结。在那之后,由于其独特的界面激子效应和优异的光伏相应特性,包括石墨烯/六方氮化硼,石墨烯/过渡金属硫化物,过渡金属硫化物/过渡金属硫化物在内的多种二维材料异质结引起了大量研究者的关注。

%//二维材料异质结
由于机械剥离和转移技术的成熟,将二维材料进行纵向堆叠而成的纵向异质结首先进入研究者的视线。2010 年,研究者通过机械剥离的方式的将单层石墨烯转移到二维六方氮化硼上,形成了具有原子层级厚度的异质结\citing{RN959-2010}。从那时起,包括 graphene/h-BN,graphene/TMDs,TMDs/TMDs 在内的大量纵向二维异质结引起了研究者的广泛关注\citing{RN309-2015, RN384-2015, RN319-2017, RN383-2012, RN368-2017},包括界面激子效应,谷电子效应和光伏响应等优异的物理性质也被逐一从二维材料异质结之中发掘出来。
例如,六方氮化硼作为二维材料中的绝缘体,在具有较大的带隙的同时也保有二维材料高质量面内结构的特点。趋近完美的二维晶体结构导致了六方氮化硼在禁带内具有极低的缺陷态密度以及较高的击穿电压,使其能够在隧穿器件中作为非常好的势垒材料。而最早发现的二维材料,石墨烯,其独特的电子结构使得可以通过引入外置栅极的方式对费米能级和态密度的位置进行调控,以此来操纵穿过势垒的隧穿电流的大小,适合用作隧穿器件中的源极和漏极材料。
迄今为止,已经有多种二维异质结通过实验合成\citing{RN961-2018, RN960-2018, RN351-2014}并用于制造场效应管\citing{RN386-2012, RN387-2012}、存储器件\citing{RN389-2013, RN388-2011}、光电器件\citing{RN390-2013}等。二维异质结使得功能更强大、特性更新颖的电子器件的出现成为了可能。
%//TODO


%//TODO 隧穿器件
不仅如此,利用六方氮化硼高耐压的特性,将石墨烯和六方氮化硼进行堆叠,能够制成具有超薄介电层的电容器。超薄的介电层能够将小至单位电荷的变化情况传导到电极之中,使之成为量子电容器。量子电容器可用于测量量子输运体系中极其微小的电荷转移情况。运用类似的思路,将石墨烯、过渡金属硫化物以及黑磷等材料纵向堆叠成类三明治结构所制得的量子电容器也被大量研究者所关注。
%//TODO 光电器件 
%//TODO https://sci-hub.se/https://onlinelibrary.wiley.com/doi/abs/10.1002/aelm.202001174

%//TODO 催化器件?



\section{二维材料的生长制备和机理研究}
\subsection{非极性二维材料}
%%非极性二维材料
\subsection{极性二维材料}
%%极性二维材料
\subsection{生长工艺及机理模型}
%//TODO https://sci-hub.se/10.1038/s41570-016-0014

\section{二维材料异质结的生长制备和机理研究}
%%纵向二维异质结
%%横向二维异质结
\subsection{纵向堆叠二维材料异质结}

\subsection{横向堆叠二维材料异质结}
横向二维异质结独特的结构使之具有许多出色的性能,在应用方面具有广阔的
前景。它们的电子特性也得到了广泛研究[29,38]。利用 MoS2-NbS2 横向异质结中的交错禁
带和弱耦合状态产生的传输间隙,可以制造出开关电流比为 106~107,漏电流约为 108
uA
的场效应晶体管[25];利用其纳米尺寸和多组分光学性质,横向 TMD 异质结可以用作单分
子探测以及制造具有可调光响应的纳米器件[39];基于 WSe2-WS2 横向异质结的 p-n 结二极
管和光电二极管,表现出了良好的整流特性并能产生较大的光电流[40];这些独特的优异
性能使得横向二位异质结为制造更高性能互补逻辑电路、高频器件、及光电探测器等电子、
光电子器件带来了新的可能。同时,由于横向二维异质结中普遍存在的自旋过滤效应,进
一步推动了自旋电子学的发展。例如 ZGNR/g-C3N4 异质结和 h-BN/zigzag graphene 构建的
异质结,都有着极高的自旋过滤效率[41]。自旋电子学利用电子的自旋特性,通过操纵电子
自旋来进行信息处理。由于其可编程逻辑元件和非易失性信息存储器件等不同领域的应用
而备受关注。随着微加工技术和大规模集成电路的发展,电子器件尺寸的缩小,自旋电子
比传统的依靠电荷传播数据具有极大的优势,如数据处理更快、能耗更低、集成度更高、
稳定性更好等。这使得具有自旋输运特性的体系在未来的数据存储,数据传输以及数据处
理方面都有着极大的发展潜力[42]。

\subsection{生长工艺及机理模型}
对于纵向二维异质结,除了机械剥离法外,在已经制备的二维异质结中,现有的制备方法有图形再生长
(patterned regrowth)[45]、刻蚀再生长(etching regrowth)[46]、化学转化法(chemical 
conversion methods)[47]、一步法化学气相沉积(一步法, one-step CVD methods) [48,49]
和两步法化学气相沉积(两步法, two-step CVD method) [43,50]等。

对于横向二维异质结,最为常用的合成方法是 CVD 法。横向的石墨烯和 hBN 异质结,
最先在 2010 年被 Feng Liu, Pulickel M. Ajayan 等人采用一步法合成出来。这个单层结构的
异质结在空间上是随机分布的,并且其带隙不同于纯石墨烯和 hBN[51]。在 2012 年,复合
石墨烯和六方氮化硼薄层通过两步法合成[52],但是其形状和边缘难以控制。A. T. Charlie 
Johnson 等人采用了常压 CVD 法合成了有笔直界面,且六方氮化硼沿着石墨烯晶体取向生
长的二维异质结,其界面的清晰程度可以通过控制生长条件进行调控[53]。An-Ping Li,Gong 
Gu 等人采用两步法合成了石墨烯和六方氮化硼的异质结,通过对于第一步中合成的石墨
烯层边缘的进行氢原子刻蚀,实现了 zigzag 形态的界面[54]。
随着更多二维材料的涌现,过渡金属硫化物,特别是 MoS2 和 WSe2 受到了广泛关注。
2014 年,Xiangfeng, Duan 等人通过横向外延的方式,用一步法制备了具有渐变无隙界面的
WS2/WSe2 和 MoS2/MoSe2 异质结并且得到了很好的电子和光电特性[55]。Wu Zhou 和
Pulickel M. Ajayan 等人采用一步法同时合成了横向和纵向异质结[3]。除了 CVD 方法外,
MoSe2/MoS2 异质结也通过电子束光刻的方式合成[56]。
由于一步合成的方法基于材料的自组装,我们很难控制二维异质结的形状和尺寸。同
时对于 WX2 体系,因为金属和硫族元素同时发生了变化,一步法很难生长如同 WSe2-MoS2
这样的异质结结,两步法的提出克服了这些困难。2015 年,有原子级的清晰界面的
WSe2/MoS2 异质结通过两步外延生长方法成功合成[57]。 同时,Gong 等人设计了一个两
步生成纵向和横向 WSe2/MoSe2 异质结的方法,能产生 169um 的异质结,且交叉污染相比
之前的一步法要小很多。Cai 等人设计了一种新颖的采用离子交换方式的两步生长方法。
采用这种方法,MoS2/ WSe2 异质结尺寸能长到 100um[58]。
厚度调制的二维异质结也能采用 CVD 方法进行制备。Zhang 等人,制备了单层和双层
阶梯状的异质结[59]。He 等人采用了 CVD 方法得到了了不同层数的 MoSe2 结。这对于层
数控制的大尺寸异质结生长具有参考价值[60]。
同时,图案化再生长和化学转化法也被用于合成二维异质结。在 2012 年,Mark P 等人
使用了两步法合成了具有空间调制效应的六方氮化硼/石墨烯的异质结。他们先用 CVD 方
法合成了石墨烯层,然后将石墨烯刻蚀形成特定图案,然后实现了 h-BN 层的空间选择性
的生长[45]。Liu 等人,采用了两步 CVD 法合成了一个具有可控形状、清晰界面、大尺寸
的六方氮化硼/石墨烯的异质结[61]。同时,MoSe2/MoS2 异质结也可以通过电子束光刻的方
式合成。这种方法能够提供清晰界面,并且能够对图案和生长过程进行控制[56]。
二维异质结也能使用物理气相转移,机械方法或者光刻等方法进行制备。Huang 等人
采用气-固生长方法合成了一个无缝的高质量的横向 MoSe2–WSe2 异质结。并具有很好的
光致发光特性[48]。Tosun 等人通过机械法合成了一个单层和多层的一型异质结。这种通过
厚度调制的二维异质结设计过程非常简单,不用涉及复杂的化学过程[62]。Jamilpanah 等人
采用了双极电极沉积技术合成了一型和二型异质结。这个方法可以在室温下进行,并且成
本低廉[49]。在这些生长方法里,图形再生长法在制备上显得较为繁琐;刻蚀再生长、化学
转化法则是利用化学反应对第一层单层二维材料薄膜进行刻蚀再生长或者直接转化,由于
化学反应的复杂性,使得我们不能很好对二维异质结的生长进行较为精细的控制。而一步
法虽然简单易操作,但是目前使用该方法制备出的二维材料异质结都在几十微米范围以内
[3,63] ,并且使用一步法制备的二维异质结在两种二维材料交接处容易产生交叉污染或形
成合金相,较难得到清晰的界面 [3,50] 。


对于纵向二维异质结,由于是层叠结构,异质结的生长与三维薄膜生长较为相似,层
与层之间的作用力由化学键变为范德瓦尔兹力,我们可以利用三维薄膜生长模型进行大致
地描述。考虑被沉积物自由能(γad),衬底表面自由能(γsub)以及界面自由能(γinter)
之间的竞争作用。当沉积物与衬底之间有很强的作用力时,γad + γinter < 𝛾𝑠𝑢𝑏,表现出层
状生长;当被沉积物与衬底表面之间的相互作用较弱的时候,被沉积原子的能量和界面的
能量之和大于或近似等于衬底表面的自由能,也就是γad + γinter ≥ 𝛾𝑠𝑢𝑏,此时表现出岛状
生长或者是层状-岛状生长的模式。对于纵向二维异质结,较弱的层间作用使其在生长过程
中一般表现出岛状(VW)的生长模式,但在掺杂和缺陷存在的情况下可能会转变为层状
(FM)或者层状-岛状(SK)的生长模式。[64]
\section{多尺度生长机理}
\section{本论文研究的主要思路及内容}