% LTeX: language=zh-CN
% TODO LIST
% 第一章 绪论
% 二维材料及其异质结
% 背景
%% 二维材料
%% 二维材料异质结
%% 实验现状
%% 理论现状
\chapter{绪\hspace{6pt}论}

\section{研究工作的背景及意义}
自从石墨烯被发现以来,二维材料由于其独特的电子结构、极强的声光耦合、奇异的层间作用、多样化的性质调控手段,已经引起了大量研究者的关注。至今为止,已经由包括石墨烯在内的多种体系的二维材料被理论上预言并在成功的实验中制备,其中包括与石墨烯结构相似的六方氮化硼(h-BN),黑鳞(Black phosphorus),过渡族金属硫化物(Transition metal dichalcogenides, TMDS)以及二维三五族半导体(III-V component semiconductors)等。这些二维材料由不同的元素组成,因此也具有各自截然不同的物理性质。


\section{国内外研究现状}