\chapter{极性锑化铟的生长机理研究}
\section{引言}
    III-V族化合物半导体被认为是下一代电子器件和光电子器件有利的材料候选。以III-V族化合物半导体\cemb{InSb}为例,块体状态的\cemb{InSb}具有较小的带隙和极高的电子迁移能力\citing{RN919-2019, RN897-1984, RN898-2020}。当结构尺度减小至低维,\cemb{InSb}展现出了更多新颖的物理特性,如非常强的自旋-轨道作用\citing{RN887-2021, RN922-2010, RN924-2015},较大的朗德g(Landé g-factor)因子\citing{RN925-2008}以及二维电子气等\citing{RN899-2020, RN926-2021}。这些新颖的物理特性使得低维\cemb{InSb}纳米结构可以用于自旋电子\citing{RN927-2017, RN928-2020},拓扑量子计算等量子器件的构建\citing{RN946-2017, RN737-2015, RN921-2016, RN933-2015}。同时,III-V族化合物半导体的单层化也有研究者进行了稳定性以及电子特性研究\citing{RN918-2013}。

    在\ref{cap:石墨烯的生长机理研究}中,我们计算了石墨烯在化学气相沉积环境下的生长机理。以石墨烯为代表的单元素平面二维材料,在理想的晶格状态下所有原子都处于同一平面,具有平面外方向的镜面对称性和反演对称性\citing{RN664-2017, RN903-2021}。而对于晶格结构以闪锌矿(zinc blende,ZB)和纤锌矿(wurtzite,WZ)为主的III-V族化合物半导体,其晶格对称性决定了在<111>晶向缺少反演对称。反演对称性的破缺使得III-V族化合物半导体的(111)晶面具有两个不同的端面,即以III族元素为端点的III极性和以V族元素为端点的V极性。不同极性表面原子的不同导致了III-V化合物半导体不同极性的(111)面表现出不同的物理化学特性\citing{RN910-2004}。III-V化合物半导体的表面极性同时也对生长过程中的III-V化合物的生长机理以及生长形貌产生影响\citing{RN929-2015, RN916-2018}。先前的研究表明,\cemb{InSb}(111)表面的许多新奇的物理现象与所处的极性息息相关。例如在特定极性的\cemb{InSb}(111)表面可以异质外延生长具有优异电子性质的新型材料 \citing{RN902-2016, RN852-2021, RN901-2019, RN900-2017, RN891-2019}。

    分子束外延(molecular beam epitaxy,MBE)技术以及各种高级材料观测技术(如球差矫正透射电子显微镜)的发展使得研究者能够进一步研究III-V化合物半导体生长过程中极性的演化及调控手段。1994年,研究者发现在\cemb{InSb/Sn/InSb}异质结构中,底端\cemb{InSb}的极性可以透过5原子层的\cemb{Sn}薄膜,对上层的\cemb{InSb}的极性产生影响。随后的研究发现,大多数的III-V化合物半导体倾向于生长V极性\citing{RN864-2019, RN930-1998, RN931-2010}。通过对实验参数和生长环境进行细致的控制,研究者能够利用分子束外延的方法对所合成的III-V化合物半导体纳米结构进行生长极性控制 \citing{RN913-2016, RN858-2019, RN889-2011, RN911-2019}。得益于合成技术的发展,研究者对于III-V半导体的生长机制进行了大量的实验观测和理论探究,力图对其中的极性演化规律产生更深的理解,从而能够更好的对III-V化合物半导体低维纳米结构进行定极性生长\citing{RN878-2020, RN940-1979, RN936-2002, RN886-2021, RN932-2018, RN894-2012, RN934-2018, RN941-2016}。

    在本章中,以III-V族化合物半导体\cemb{InSb}为例,我们系统的探究了双层\cemb{InSb}在\cemb{Bi}(111)衬底上的生长序列以及形貌演化规律。我们的研究表明在\cemb{Bi}衬底上,\cemb{InSb}的极化从第二层生长开始。单层的\cemb{InSb}在在\cemb{Bi}衬底上表现出非晶的形态。我们绘制了双层\cemb{InSb}在
    \cemb{Bi}衬底上的极化相图,并且探究了从单层到双层\cemb{InSb}表面极性变化的物理成因。
\section{计算细节}
    在本章中,密度泛函理论主要使用 Vienna ab-initio Simulation Package (VASP) 软件包进行计算\citing{RN681-1996, RN682-1996}。在密度泛函理论计算中,我们使用广义梯度近似(GGA)下的 Perdew-Burke-Ernzerhof (PBE)泛函描述电子之间的交换关联作用\citing{RN683-1996}。平面波的截断动能取为为$\SI{500}{\electronvolt}$。\cemb{InSb}与衬底\cemb{Bi}之间的范德瓦尔斯作用使用Grimme的DFT-D2方法进行描述,并带有Becke-Johnson阻尼作用 \citing{RN937-2010, RN938-2011}。


\section{单层锑化铟的生长机理}
    \subsection{单层锑化铟的生长过程}
    \subsection{单层锑化铟的极性演化}
\section{双层锑化铟的生长机理}
    \subsection{双层锑化铟的生长过程}
    \subsection{双层锑化铟的极性演化}
    \subsection{双层锑化铟的极性演化机理}
    \subsection{双层锑化铟的极性演化相图}
\section{总结}